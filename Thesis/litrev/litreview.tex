\chapter{Literature Review}

Hubel and Wiesel, in their seminal paper in 1962, described the receptive field (RF) properties and the functional organisation of neurons in the primary visual cortex (V1) of cats. They showed that most cortical neurons were sharply tuned to orientation, showed ocular dominance and could be classified as simple or complex. They proposed a hierarchical, feedforward model that gave rise to orientation selectivity, simple cells as well as complex cells. Further, they proposed that neurons in V1 were grouped into columns based on the optimum orientation of the neurons. Since their first report, the theories proposed by Hubel and Wiesel have been heavily interrogated with evidence presented both to support and oppose their proposals. Some of this literature is reviewed below.

\section{The Visual Pathways}

Information in the visual system flows from the retina to the dorsal lateral geniculate nucleus (dLGN; referred to henceforth as the LGN) to the primary visual cortex (V1) From V1, visual information is further transmitted to extra-striate visual cortices. An alternate pathway to the extra-striate visual cortices also exists. In this section, a short description of the receptive field properties that are studied in this thesis as well as their transformation at each stage of the visual pathway are described.


\subsection{Receptive field Properties}
The receptive field of a neuron "Spatial distribution of responsiveness as mapped literally in the external visual field" (Levick, 1996). 


Hartline (1940) first described the receptive field of optic nerve fibres as the part of the visual space where the activity of the neuron can be influenced \cite{Hartline1940a}. Since their description, the receptive fields of neurons from every visual area have been studied extensively. The receptive field properties that have been studied here are:

\textbf{Polarity preference of neurons}: Starting from the retina to the striate cortical neurons, neurons or sub-regions of neurons respond to light increments (on stimuli) or light decrements (off stimuli). Care must be taken to not interpret these responses as excitation and inhibition respectively. Both on and off stimuli could produce an excitatory response on neurons and can mutually inhibit the other response.

\textbf{Orientation Tuning}: Neurons that selectively respond to edges of certain orientations were first reported in the primary visual cortex in cats \cite{Hubel1962d}. Since its first report, neurons in most regions of the visual pathways have been shown to possess some degree of orientation selectivity.

\textbf{Spatial Frequency Tuning}: Based on the organisation of the on and off subregions, a neuron responds selectively to certain spatial frequencies of the stimulus.

\textbf{Linearity of Spatial Summation}: Within its sub-regions, a linear neuron summates linearly and between the different summation, there is inhibition so that responses of the neuron can be predicted from its receptive field structure.

\subsection{Retina}

The retinal ganglion cells (RGCs) are  the first stage of neural processing in vision. Most of the features observed in the primary visual cortex are already present to some extent in the retinal ganglion cells. In this part, the receptive field properties of the retina are described. Most of these receptive field properties are common to cats, macaques and tree shrews, the three species studied here. Any significant differences between the three species are pointed out.

\paragraph{Receptive Field Structure}

RGCs have concentric centre-surround receptive field structure with neurons either showing a preference for light increments or light decrements \cite{Kuffler1951}. In neurons, the centre responded to light increments while the surround responded to light decrements (on-centre-off-surround neurons) and in off neurons, the centres responded best to light decrements while the surround responded best for light incremements (see fig \ref{fig:rforg}). Later studies demonstrated,  both electrophysiologically and morphologically that RGCs are a diverse group of neurons with neurons. Among these, neurons that have both on-off representation in the receptive field centres (Cleland et al., 1971; Cleland \& Levick, 1974; Stone \& Fukuda, 1974) and those that have three concentric regions --- a centre that was on-off, an immediate surround that was sensitive to off stimuli and third region sensitive to on stimuli--- have been described. However, these neurons form a minority and most neurons show the centre-surround organisation described above.

\begin{figure}[H]
	
	\includegraphics[width=0.5\linewidth]{litrev/retinarf.jpg}
	\centering
	\caption{The different receptive field organisations in the visual system. a) concentric receptive fields- off centre/on surround and on centre/ off surround neurons. receptive fields are slightly elongated. b) cortical simple cells with segregated on and off regions. c) cortical complex cell where on and off stimuli evoke a response at every part of the rf.}
	
	\label{fig:rforg}
\end{figure}

\paragraph{Orientation Tuning}

While it was originally thought the retinal neurons were untuned for orientation, Hammond showed that the RGC receptive fields were often elliptical \cite{Hammond1974}. Following this, Levick and Thibos showed that RGC neurons indeed demonstrated small orientation biases, especially at higher spatial frequencies and that these neurons were predominantly biased for the radial orientation \cite{Levick1980, Levick1982c}. The radial orientation was the orientation of the line joining the centre of the receptive field to the centre of foveal representation. They also found a bias for the tangential orientation (orthogonal to the radial orientation) when alternating gratings were used \cite{Thibos1985}. The orientation biases of RGCs was attributed to the elongated dendritic fields of RGCs \cite{Leventhal1983a}. Later studies showed that the orientation selectivity of RGCs involve complex interactions with an extended surround \cite{Shou2000}.

\paragraph{Spatial Frequency Tuning}

Enroth-Cugell and Robson first suggested that the spatial frequency tuning of RGCs could be modelled by a difference of Gaussian function, with $\sigma$s equal to the radius of the center and the surround of retinal receptive fields \cite{Enroth-cugell1966b}. It has since been shown that RGCs have low pass spatial frequency tuning, that the peak spatial frequency selectivity of RGCs decreases as eccentricity increases and that at higher spatial frequencies, RGCs show orientation selectivity \cite{Levick1980, Levick1982c}. Cat RGCs respond to spatial frequencies from 0 to 1-5 cycles/deg (ref).

\paragraph{Linearity of Spatial Summation}

Enroth-Cugell and Robson (1966) also classified RGCs into X and Y cells based on the linearity of spatial summation they demonstrated within their receptive fields. When shown light spots of increasing sizes, X cells showed responses that indicated linear summation over their receptive fields whereas Y cells did not. A more descriptive classification was suggested by Cleland and colleagues who classified retinal neurons into sustained and transient neurons which corresponded well with the X and Y neurons respectively (Cleland et al., 1971). X and Y cells could both be on and off centred. Y cells usually had larger receptive fields (Linsenmeier et al., 1982). Functionally, X cells have been associated with transmitting colour information while Y cells have been associated with transmitting movement information. W cells, which are said to transmit blue cone information are also present within the retina (Ref) and showed non-linear responses to stimuli. X and Y cells showed 'brisk' response to stimuli while W cells showed 'sluggish' response to stimuli. X cells showed a higher spatial frequency cut-offs when compared to Y cells \cite{Thibos1983}.


A wide variety of RGCs have been described on the basis of morphology and function, with over 22 different types reported (Ref). In the following review, the receptive field structure, the orientation tuning, the spatial frequency tuning and the linearity of spatial summation are described as these are the properties that are most relevant for this thesis. Other prominent features of RGCs include colour and direction selectivity (Ref).

\subsection{The Geniculo-Striate Pathway }

The primary target of retinal neurons is the dorsal lateral geniculate nucleus (dLGN; referred to as LGN throughout this thesis) which then projects to the primary visual cortex (V1) forming the geniculo-striate pathway. While it was thought the the LGN was a relay area, the morphological and electrophysiological properties of its neurons as well as its projections from the visual cortex, brainstem nuclei and perigeniculate nuclei and the presence of inhibitory interneurons within the area has drastically changed this view (see Sherman and Koch, 1981 for review). The receptive field properties of these structures are discussed below.

\subsubsection{Dorsal Lateral Geniculate Nucleus (LGN)}

The LGN receives direct, feedforward input from the retina but also feedback inputs from V1 and the brainstem areas as well as inputs from geniculate inhibitory neurons. While the inhibitory and feedback inputs account for the majority of connections to the geniculate neurons, the feedforward inputs account for the majority of the LGN neurons' receptive field properties. 

\paragraph{Receptive Field Organisation}
As in the retina, the majority of LGN neurons have centre-surround organisation and are either on-centre/ off-surround neurons or off-centre/ on-surround neurons \cite{Hubel1961}. Further, an extra-classical receptive field (ECRF) is also observed in most LGN neurons (ref). While the centre is meant to reflect the feedforward retinal properties, the ECRF is meant to be due to the inhibitory inputs and cortical feedback (ref).


\paragraph{Orientation Tuning}
Similar to RGCs, LGN neurons were first considered to be untuned to orientation. However, it was later shown that LGN neurons were indeed tuned to orientation \cite{Xu2002} and at higher spatial frequencies, this orientation tuning was more prominent \cite{Vidyasagar1982, Vidyasagar1984b, Xu2002, Suematsu2012}. This orientation selectivity has been attributed to a combination of elongated receptive fields and surround suppression \cite{Suematsu2012, Naito2013}.  Studies  in the kitten LGN showed that the neurons were already biased for orientation at the early stages of development, indicating that there may be a morphological basis for orientation tuning in the LGN \cite{Albus1983}. It was suggested that the orientation biases observed in the LGN were a reflection of the orientation biases in the retina \cite{Soodak1987} however, studies have shown that removing the inhibitory input to the LGN neurons reduces the orientation selectivity of these neurons \cite{Vidyasagar1984}. The ECRF of LGN neurons have also been shown to have broad orientation biases which do not always co-incide with the optimum orientation of the CRF \cite{Sun2004,Naito2007}. The majority of the LGN neurons were tuned for the radial orientation \cite{Shou1989, Smith1990b}.

\paragraph{Spatial Frequency Tuning}
Retinal ganglion cells were tuned to higher spatial frequencies than LGN cells (Maffei \& Fiorentini, 1972). Spatial frequency tuning of LGN neurons also tended to reflect the spatial frequency tuning of retinal neurons \cite{So1981}. ECRF tuned to lower spatial frequencies \cite{Sun2004}. Y cells were better tuned at lower spatial frequencies while X cells fired at higher spatial frequencies \cite{Lehmkuhle1980}.

\paragraph{Linearity of Spatial Summation}
On Centre retinal X cells project to on centre LGN X cells- Parallel pathways from retina to LGN to cortex (Cleland et al., 1971). Magno- parvo pathways- similar to the Y/X dichotomy but not needed really. Something about the parallel pathways X cells were tuned to higher spatial frequencies, had higher spatial frequency cut-offs but lower overall firing rates compared to X-cells (Cleland et al., 1971; Derrington \& Fuchs, 1978). X, Y and W cells== Parvo, Magno and Konio? But parvo layers also have Y cells. So not exactly. X cells- smaller receptive field sizes, slower transmission speeds- Forms part of a slow pathway from the retina to the LGN--- then to the cortex? Y cells- lower SF resolution than X cells (Saul \& Humphrey, 1990).

\paragraph{Ocular Dominance}  

In the LGN have to deal with ocular dominance. Arranged in layers within the LGN. There is some mixing? Serve to provide inhibition (Xue et al., 1987)

\subsubsection{The Primary Visual Cortex (V1)}

hubel and wiesel- described receptive fields in the primary visual cortex. Studied a lot. Neurons are orientation selective

X-Y dichotomy= simple/complex dichotomy in cortex? receptive field size and stimulus speed size are similar- Hoffmann et al., 1971.

\cite{Stone1973} X cells project almost exclusively to area 17 in cats, Y cells projections bifurcate and project to area 17 and 18. 

\paragraph{Receptive field organisation}

Separate on and off sub-regions- simple cells
Mixed on and off sub-regions- complex cells
End-inhibited- hypercomplex cells- proposed an heirarchical model where a simple cell receives from on and off centre LGN neurons, a complex cell receives inputs from many simple cells and a hyper-complex cell receives inputs from many complex cells. 
Complex cell rfs are bigger than simple cells
Majority of simple cells in layer 4 and layer 2/3 cells are in layer 2/3 but also layer 4 has complex cells.

\cite{Kelly1974} Simple cells are mostly layer 4 stellate cells, hypercomplex cells are layer 2/3 pyramidal cells, complex cells are layers 2/3/5/6 pyramidal cells. 

\cite{Leventhal1978} Cats Simple cells- lower relative degree of binocularity, more selctivity for stimulus orientation, have smaller receptive fields, lower peak responses, lower spontaneous activity.

\cite{Wilson1976a} Cats Proportion of simple cells relative to complex cells rose with eccentricity. Cells had larger receptive fields, less orientation selectivity and higher preferred speeds at increasing eccentricity but more obvious in complex cells
\begin{figure}[H]
	
	\includegraphics[width=0.5\linewidth]{litrev/corticalrf.jpg}
	\centering
	\caption{Examples of simple and complex cells in V1. The simple cells are symmetric on and off centre cell. Anti-symmetric neurons where there are an even number of antagonistic sub-regions can also be found in V1. The complex cell responds well to either light or dark stimuli at any location over its receptive field.}
	
	\label{fig:rforgV1}
\end{figure}

\paragraph{Ocular Dominance}
In V1, neurons show a wide range of ocular dominances. Bell curve. Organised in columns.

\paragraph{Orientation selectivity}

While LGN neurons retain the properties observed in the LGN, layer 4 of the cats is where there is an extensive transformation of receptive field properties for the first time. Unlike lateral geniculate nucleus neurons which respond to wide field, circular flashing stimuli, V1 neurons respond best bars and edges of the optimal orientation \cite{Hubel1962d}. Since then, orientation selectivity has been detected in the primary visual cortex of all species that were studied. In the V1, the distribution of orientation selectivities differ between the macaque, tree shrews and cats. While in the cats, orientation selectivity is observed in the cortical input layer (layer 4), in the macaques and tree shrews, layer 4 neurons show broad orientation biases as described in the LGN. In all three species, supragranular layers are tuned to orientation. Studies that examined the distribution of orientation selectivities also showed that while overall layer 4 neurons showed broader orientation tuning and layer 2/3 neurons showed sharper orientation tuning, individual neurons in each of these layers show a wide range of orientation selectivities in macaques and tree shrews. Simple cells showed sharper orientation tuning than complex cells \cite{Henry1974, DeValois1982, Leventhal1978}


\cite{DeValois1982}: Macaques neurons in V1 are tuned to orientation and direction. Different degrees of orientation selectivity within simple/complex cells. Within foveal neurons, vertical and horizontal orientations are better represented. Most inhibition at orientations adjacent to the orthogonal orientation rather than at the orthogonal orientation- models.

\cite{Fregnac1978} kittens- sharp orientation selectivity from 12 days already but ocular dominance takes some time. Also binocular inputs required to generate the full range of orientations preferences in kittens. kittens under 3 weeks of age a higher proportion of binocular neurons. Horizontal and vertical biases in early stages of development.

\cite{Sherk1975} Also kittens- sutured eyes= recorded at 22-23 days- already tuned to orientation. 

\cite{Cynader1975} Cats reared in a room with unidirectionally moving stimuli- direction selectivity- direction of the moving grating (leftward). Most units were oriented vertically. No such effect in the SC. 

\paragraph{Spatial Frequency Tuning}

\paragraph{Linearity of Spatial Summation}

Other features that are present in neurons- contrast invariance, length tuning and temporal frequency tuning- should mention these- explain the phenomena

\cite{Skottun1991}: Cats- did modulation ratio and found that simple/complex cells are classified well uaing the modulation ratio (F1/F0). Bimodal distribution.

\cite{Ferster1986}: Cats- intracellular recordings, found that inputs were already sharply tuned to orientation- not sharpened in the cortex. 

\section{Parallel pathways in the primary visual cortex}

In most species, visual information is segregated on the basis of their functional properties into different pathways. In cats, macaques and tree shrews, the functional segregationdiffers. These differences are briefly examined below (also see Figure 1 for summary).

In macaques, chromatic and achromatic information is segregated in different pathwaysin their projections from retina to LGN to V1. The magnocellular pathway (M-) transmits achromatic information and the neurons in this pathway respond to luminance changes (Hicks et al., 1983; Kaplan et al., 1990; Dacey, 2001). The parvocellular (P-) (Hicks et al., 1983; Kaplan et al., 1990; Merigan \& Maunsell, 1993) andkoniocellular (K-) (Dacey, 2001; Roy et al., 2009)pathways transmit chromatic information. The  major targets of these projections in macaques are in layer 4Cα, 4Cβand layer 3B of V1 (Casagrande \& Kaas, 1994). This segregation was believed to be maintained even in extrastriate areas (Bullier \& Henry, 1980; Casagrande \& Kaas, 1994). However, there is evidence to suggest that there is considerable overlap in inputs as early as layer 4 (Casagrande \& Kaas, 1994; Callayway, 1998; Vidyasagar et al., 2002).

In comparison, LGN inputs to V1 in the tree shrew are segregated into ON, OFF and W-cell pathways (Conway \& Schiller, 1983; Conley et al., 1984; Holdefer \& Norton, 1995). ON cells respond to increases in luminance and OFF cells respond to decreases in luminance. The ON, OFF and W-cells terminate in layers 4A, 4B and 3C of V1 respectively (Conley et al., 1984). Layer 4A mostly have on neurons and 4B, mostly off neurons (for review, see Fitzpatrick, 1996).

In cats, the inputs to V1 segregate differently. X and Y cells of the LGN project to layers 4C and 4A+B respectively (Wilson et al., 1976; LeVay \& Gilbert, 1976). X-cells show a sustained response when presented a stimulus. They also sum signals linearly within their receptive fields. That is, when presented with dark and light stimulus regions over the receptive field at the appropriate phase, there is virtually no response as the cell sums the signals from the ON and OFF sub-regionslinearly. Y cells on the other hand sum non-linearly within their receptive fields and they also have a transient response when a stimulus is presented, irrespective of phase (Enroth-Cugell \& Robson., 1966).
While there are major differences in physiological properties of the different pathways, some similarities have been found. For example, it has been shown that there is some extent of on/off segregation as observed in the tree shrew within the parvocellular layers of the macaque LGN (Schiller \& Malpeli, 1978). It was also originally thought that parvocellular cells were X-cells and magnocellular cells were Y-cells (Dreher et al., 1976). However, this is not entirely the case. While most P-cells are indeed X cells, 75\% of M- cells are also X-cells in macaques (Shapley et al., 1981). Similarly, most neurons in the tree shrew LGN are also X cells, with cells showing non-linear summation only observed in 2 of the 6 layers (Conway \& Schiller, 1983).


\section{Mechanisms underlying feature selectivity}

\subsection{Models of Orientation selectivty}

Of all the findings reported by Hubel and Wiesel, perhaps the most striking feature of cortical cells is the orientation selectivity of the neurons. While recording from the primary visual cortex, H\&W reported that the neurons preferred edges. Further, they found that the neuron only preferred edges of a very specific orientation often not responding at all to orientations 90$^o$ apart. This was in stark contrast to RFs found in the retina and lateral geniculate nucleus (LGN) neurons that showed poor orientation selectivity. H\&W, in order to explain the appearance of orientation selectivity, proposed the excitatory convergence model of orientation selectvity.

\subsubsection{The Excitatory Convergence Model} Hubel and Wiesel suggested that the orientation selectivity of the cortical neurons came from the excitatory convergence of inputs from circular LGN receptive fields which were arranged in a row. The cortical neuron would then be tuned to the same orientation as the line joining the centres of LGN receptive fields (Figure \ref{fig:HW}a). When the LGN neurons encountered an edge of the optimum orientation, then the centers of all the LGN neurons are simultaneously activated and hence the cortical neuron gives a strong response. When an edge of the non-optimum orientation is presented however, the LGN neurons fire sequentially and as a result, the cortical neuron gives a weaker output (Figure \ref{fig:HW}b) giving the cortical neuron orientation selectivity.

	\begin{figure}[H]
	\centering
	\includegraphics[width=0.8\linewidth]{litrev/hw1962adapt.jpg}

	\caption{The model of excitatory convergence proposed by Hubel and Wiesel (1962). a) LGN neurons with unoriened receptive fields project to Layer 4 neurons in the primary visual cortex. The receptive field of the resulting V1 neuron is elongated along the axis in which the LGN rfs were organised. b) The inputs that the LGN neurons provide to layer 4 neurons: When a stimulus of the optimum orientation is presented, the LGN neurons fire together and as a result, the cortical inputs at this orientation is better (solid line) than the input when a stimulus of the orthogonal orientation was presented (dotted line). Adapted from Hubel \& Wiesel, 1962 and Priebe, 2016.}
	\label{fig:HW}
	\end{figure}

\paragraph{Evidence for the excitatory convergence model}

The excitatory convergence garnered a lot of support and is still one of the most popular models of orientation selectivity. One of the first pieces of evidence that led to the proposal of such a model is the length summation of neurons. Hubel and Wiesel found that neurons in V1 responded to stimuli of increasing lengths, showing length summation in response to bars \cite{Hubel1962d, Gilbert1977, Rose1977}. Hubel and Wiesel interpreted these results as meaning that many different LGN neurons converging on the receptive field of the V1 neuron. 

It was shown in intracellular recordings that the orientation selectivity of LGN inputs to the primary visual cortex were already sharply tuned to orientation when studied using intracellular recordings (Ferster, 1986; Ferster, 1996). When the cortical circuit was silenced using cooling (Ferster et al., 1996), using pharmacological intervensions (Nelson et al., 1994; Chapman et al., 1991) or using electrical stimulation (Chung \& Ferster, 1998; Kara et al., 2002) the orientation selectivity of the cortical neurons were still preserved.

A second line of evidence examined the organisation of LGN inputs to the V1. In the excitatory convergence model, Hubel and Wiesel suggested that the inputs of the cortical neuron came from LGN neurons that are arranged in a row. Cross correlation and simultaneous multi-electrode cross correlation studies showed that the off and on sub-region of simple cells received inputs from on and off geniculate neurons (Tanaka et al., 1983; Reid \& Alonso, 1995, Jin et al., 2011) Further, studies showed that the receptive fields of the input neurons were arranged in a row along the orientation of the cortical neurons as predicted by the excitatory convergence model \cite{Mooser2004a, ClayReid1995}.

Further support for the excitatory convergence model came from the studies that examined the laminar position of cortical simple and complex cells. In the excitatory convergence model, the LGN neurons projected to cortical simple cells which then projected to cortical complex cells. In line with this, several studies showed that simple cells were mostly located in layer 4 and layer 6 where the geniculate inputs terminated while complex cells tended to be sparsely represented in layer 4 of the cortex (Crowder et al., 2007; Gilbert, 1977, Gilbert \& Wiesel, 1979; Hirsch et al., 1998a; Hirsch et al., 1998b; Kelly \& Van Essen, 1974; Martinez et al., 2002; Martinez et al., 2005; Ringach et al., 2002).

\paragraph{Evidence against the excitatory convergence model}

While plenty of studies have been undertaken to support the excitatory convergence model, the inabilty of the excitatory convergence model to explain the contrast invariance \cite{Sclar1982} and the role inhibition plays in sharpening orientation selectivity \cite{Creutzfeldt1974, Sillito1975}. These phenomena are detailed below.
 
\paragraph{Contrast Invariance of orientation selectivity}

Neurons in the primary visual cortex showed contrast invariant orientation tuning \cite{Sclar1982, Skottun1987, Alitto2004} i.e. when shown oriented stimuli of different contrasts, the extent of orientation selectivity of the studied neuron remained unchanged. This contrast invariance of orientation selectivity is incompatible with feedforward models of orientation tuning, as such models predict that as the contrast increases, the response of the LGN neurons that provide input increases at both the optimum and the non-optimum orientations, increasing the orientation selectivity of the neurons (Priebe and Ferster, 2012: figure 3). Inhibition maybe used to explain the contrast-invariance of simple cells. Computational models have shown that either cross-orientation inhibition \cite{Troyer1998} or orientation non-specific inhibition \cite{Ben-Yishai1995, Somers1995, Sompolinsky1997} can both be used to explain the contrast invariance of V1 neurons however, intracellular evidence suggested that there was little by way of cross-orientation inhibition in V1 \cite{Anderson2000}. Alternately it was suggested that contrast-invariant orientation selectivity could arise from neurons in LGN that had orientation biased responses \cite{Vidyasagar1982}, which provided contrast invariant inputs themselves \cite{Naito2013, Viswanathan2015}. It was suggested that the contrast invariance of neurons in both the LGN and V1 could be explained by contrast-dependent trial-to-trial variability \cite{Finn2007, Sadagopan2012, Priebe2012, Viswanathan2015}.


\paragraph{Role of inhibition}
Different lines of evidence pointed out the important role that inhibition plays in sharpening orientation selectivity. These are detained below. Most neurons not only receive direct excitatory inputs from geniculate neurons but also receive di-synaptic inhibitory inputs \cite{Creutzfeldt1968, Ferster1983}.


\paragraph{Intracellular studies}

While some intracellular studies suggested that inputs to the cortical neurons were already tuned to orientation, others showed that there were separate excitatory and inhibitory inputs to neurons observed in the form of excitatory and inhibitory post-synaptic potentials. Creutzfeldt and colleagues measured the intracellular responses of neurons and found that most neurons had an inhibitory surround several times the size of the excitatory receptive field \cite{Creutzfeldt1974}. Similarly, a case for inhibition in the cortex of awake behaving monkeys where there was feedforward excitation as well as inhibition was also made \cite{Celebrini1993}. While Ferster showed that the EPSPs of simple cells were already tuned to orientation \cite{Ferster1986}, Volgushev and colleagues studied the dynamics of orientation selectivity and found that orientation selectivity sharpened as a function of time \cite{Volgushev1995} (although see \cite{Celebrini1993}), further indicating that inhibition plays an important role in sharpening orientation selectivity.


\paragraph{Iontophoretic application of GABA- inhibitors}

Studies used intravenous and iontophoretic application of GABA inhibitors and found that the degree of orientation selectivity of a neuron was affected when inhibition was removed. Early studies used either intravenous or iontophoretic application of bicuculline (bic) and found that there was mixed effect on the orientation selectivity of cortical simple cells \cite{Pettigrew1973, Sillito1975}. Howver, when 3-mercaptopropionic acid (MP) was administered intravenously along with iontophoretic application of bic, it was shown that orientation selectivity was completely abolished in a small proportion of neurons, suggesting that a more potent GABA inhibitor was necessary \cite{Tsumoto1979}. As a result, when Sillito and colleagues used N-methyl bicuculline, a stronger GABA inhibitor, they were able to abolish orientation selectivity in 9 out of 13 simple cells, suggesting that GABA mediated inhibition formed an important role in  shaping orientation tuning of simple cells \cite{Sillito1980}. Nelson and colleagues used intracellular injections of Cesium Fluoride (CsF) and chloride-channel blockers to examine orientation selectivity without inhibition. Unlike previous results, they found that in the presence of the inhibitory blockade, the neurons were still orientation selective. However, during their experiment, they applied a constant hyperpolarising current to reduce the spontaneous activity of the neuron. It could be that they inadvertently provided the inhibition that they were trying to blockade by raising the threshold of firing, further suggesting that neurons required inhibition for orientation selectivity \cite{Nelson1994}. A recent study showed that when GABA was iontophoretically applied, untuned neurons showed orientation selectivity \cite{Li2008}.

\paragraph{Extracellular studies}

Three different types of studies where neurons were recorded from extracellularly showed that inhibition played a significant role in the orientation selectivity of neurons. First, studies that examined the linear and non-linear components of V1 neuron responses found that the orientation selectivity of neurons could not be explained by their receptive field structure indicating the presence of non-linearities \cite{Watkins1974, Gardner1999}. These non-linearities were assumed to be due to intracortical inhibition. Second, cross-correlation analysis between pairs of neurons within the primary visual cortex suggested that in pairs of neurons that were within a 1mm apart, inhibitory interactions were observed \cite{Hata1988}. Finally, studies where two super-imposed gratings of different orientations found that an orientation non-specific inhibition was observed in a majority of cases with cross-orientation inhibition present in a small proportion of neurons \cite{Bonds1989}. Studies that looked the dynamics of orientation tuning both extracellularly reported that the orientation selectivity of neurons increases as a function of time after stimulus onset suggesting that the orientation selectivity was sharpened by inhibition \cite{Shapley2003}.

\subsubsection{Alternate models of orientation selectivity}

The models proposed against the excitatory convergence model can be classified into three categories: ones that employ intracortical circuitry in sharpening orientation selectivity, those that rely on the relative arrangement of the on and off sub-regions to explain orientation selectivity and finally where cortical inputs are derived from oriented LGN neurons and further sharpened by intracortical mechanisms.

\subsubsection{1) Cross-orientation inhibition}

\paragraph{The Model} In the cross-orientation inhibition model, cortical neurons receive excitatory inputs from unoriented LGN neurons and the orientation tuning of the neurons was generated by inhibition from neurons of the orthogonal orientation, i.e. orientation selectivity was generated entirely through intracortical inhibitory mechanism. One obvious issue with this model is that it does not explain where the orientation selectivity of the inhibition comes from and would need to rely on other mechanisms to explain this orientation tuning. Keeping this in mind, the evidence supporting and against the cross orientation inhibition are described below. 

\paragraph{Evaluation of the cross-orientation inhibition}

Apart from evidence suggesting that inhibition was required to sharpen orientation selectivity generally, two lines of evidence suggested specifically that said inhibition was tuned to the null orientation of the neuron. First was evidence from extracellular studies where response to stimuli oriented orthogonal orientations was tested. Morrone and colleagues examined the respnse of cat V1 neurons showed that simple and complex cell responses were reduced when a grating of the orthogonal orientation was presented against a conditioning stimulus consisting of one-dimensional noise of the optimum orientation \cite{Morrone1982}. They also showed that such response in reduction was due to GABA mediated inhibition \cite{Morrone1987}. Similarly, in a study where Bonds used two gratings, one of the optimum orientation and the second of the non-optimum orientation drifting at different temporal frequencies, found that the mask of non-optimum orientation significantly reduced the response at the optimum orientation \cite{Bonds1989}. Second line of evidence comes from studies where the lateral network of neurons were inhibited. Eysel and colleagues applied GABA to neurons located 500-600 $\mu$m away from the recorded neurons found that the orientation selectivity of the recorded neuron widened significantly in areas 17 and 18 \cite{Worgotter1988, Eysel1990, Crook1992}. Further, it was shown that this loss of orientation selectivity was more pronounced when the silenced neurons had an optimum orientation greater than 45$^o$ away from the optimum orientation of the recorded neurons \cite{Crook1992, Crook1997}. It was also shown that the inhibitory inputs of neurons in V1 came from neurons tuned to orientations greater than 30$^o$ of the orientation of the recorded neuron \cite{Kisvarday1997}. The above studies gave strong evidence for the presence of cross-orientation inhibition in the cortex.

Strong opposition to the cross-orientation inhibition model came from studies which showed that inhibition to neurons were not tuned to the orthogonal orientation but rather to the optimum orientation of neurons \cite{Anderson2000, Roerig2002, Tan2011a}. Following this, it was suggested that the cross orientation suppression observed in V1 neurons could be explained by non-linear mechanisms of the geniculate neurons, namely spike-rate rectification and response saturation \cite{Priebe2006}, suggesting that these non-linearities along with a Hubel and Wiesel type excitatory convergence is sufficient to produce sharp orientation selectivity, even in the absence of cross-orientation inhibition  \cite{Priebe2006, Priebe2012, Priebe2016}. However, when the dynamics of the cross orientation suppression were studied, it was found that cross-orientation inhibition occurred even before excitatory response of the neurons suggesting that the origins of such inhibition maybe from feedforward inputs or fast spiking local inhibitory interneurons \cite{Smith2006}. The temporal dynamics of the cross-orientation suppression suggested that an initial suppression arises from feedforward signals \cite{Freeman2002, Li2006} followed by a second-wave of cross-orientation suppression originating from intracortical mechanisms immediately after the excitatory response starts \cite{Kimura2009}. These studies along with the fact that the orientation selectivity of inhibitory neurons cast considerable doubt on the cross- orientation inhibition model for generating orientation selectivity. Such inhibition could at best sharpen pre-existing orientation biases in the cortical inputs.

\subsubsection{2) Recurrent models}

Recurrent models are a second example of models which suggest that orientation selectivity is an emergent property of the cortical network. Recurrent model either involve recurrent excitation or a balance between recurrent cortical excitation and inhibition to explain orientation selectivity. These models are explained and evaluated below.

\paragraph{The model}

Several versions of the recurrent model exists but in they all rely on the intracortical circuitry to sharpen orientation selectivity. According to this model, once a small, oriented input excites a cortical neuron, recurrent excitation between neurons tuned to the same orientation improves orientation selectivity \cite{Somers1995, Douglas1991a, Douglas1995}.Shunting inhibition then lent stability to the network and prevented runaway excitation (Heeger 1992, 1993, Carandini \& Heeger 1994, Chance \& Abbott 2000, Douglas1995, Carandini1997, Douglas2004). One issue with the recurrent excitation model was that while the model amplified the orientation signal it received, the initial orientation selectivity still needed to  be present in the feedforward inputs \cite{Douglas1995, Vidyasagar1996b} but a later study showed that even the initial orientation bias may be generated via intracortical recurrent excitation \cite{Adorjan1999}.

\paragraph{Evaluation of the recurrent excitation model}

The recurrent excitation followed by inhibition explained may of the existing experimental evidence. Anatomical studies showed that only about 5-10 \% of cortical synapses were formed by LGN afferents with a majority of synapses being formed by intracortical afferents(Ref). Of the intracortical some 85\% of the connections were excitatory, with inhibitory connections forming only a small part of the circuitry (ref). Given that only a small excitation arises from the geniculate, the recurrent excitation serves to both amplify and sharpen the orientation response of the neurons. 

In the model proposed by \cite{Douglas1995}, a small feedforward inhibition is also present so that when a stimulus of the non-optimum orientation is presented, the small LGN excitation overlaps with feedforward inhibition and the cell remains silent. This explains the results of the many intracellular studies detailed earlier that showed that the EPSPs were  already tuned for orientation and no IPSPs were observed \cite{Ferster1986, Anderson2000}. If the original feedforward inhibition is of a small magnitude, it would not be easily visible in intracellular recordings as it would show only small changes in somatic input conductances.

Studies that examined the dynamics of orientation selectivity showed that the orientation selectivity of neurons sharpen over time and these results are consistent with the predications of the recurrent excitation model \cite{Ringach2002c, Volgushev1995, Pei1994}. These studies showed that initially neurons were broadly tuned to orientation and sharpened within a few ms of the stimuli and while intracortical inhibition is said to play an important role in sharpening the orientation selectivity, recurrent intracortical excitation has also been implicated in such sharpening \cite{Ringach2002c, Vidyasagar1996b, Pei1994, Adorjan1999}

More recent versions of the recurrent model have shown that recurrent intracortical inhibition also plays a big role in sharpening orientations selectivity. Evidence for this claim comes from studies that have shown that both the excitation and inhibitory input of a neuron are tuned to the same orientation \cite{Anderson2000, Monier2003, Tan2011a}. Such inhibitory input is also said to act by moderating the level of neuronal excitation \cite{Nelson1998, Sato2016}. The current view of the recurrent model is that sharp orientation selectivity is generated by recurrent excitatory inputs balanced by recurrent inhibitory inputs \cite{Nelson1998, Shu2003}.


While attempts have been made to address the issue of the origins of the orientation selectivity that is amplified by the cortical circuit, this question remains one of the biggest problem of the intracortical models of orientation selectivity. Orientation selectivity in the inputs need to arise in the feedforward signal from the LGN as suggested by Hubel and Wiesel. These models are reviewed below. 



\subsubsection{Feedforward Models}

\subsubsection{Spatially offset subregions}

Feedforward models are typically involve spatially offset input sub-regions. In the excitatory convergence model, such spatial offset is along the axis of the orientation. Other models where excitation and inhibition and on and off sub-regions are spatially offset are discussed below.

\paragraph{1) Spatially offset excitatory and inhibitory sub-regions}

Heggelund proposed a model of orientation selectivity based on the fact that cortical simplec cells tended to receive both excitation and inhibition from neurons of the same polarity \cite{Heggelund1981}. According to this model, an on-centred neuron received direct excitatory stimulation from an on-centred LGN neuron and inhibition from another on centred LGN neuron via an inhibitory interneuron. A spatial offset between the excitatory and inhibitory inputs theoretically led neurons to have sharp orientation selectivity. Later models of push-pull receptive field organisation have a similar organisation. In this model as well, the on sub-region receives excitatory inputs from on-centre LGN neurons, but also receives inhibition from an opposite polarity inhibitory interneuron so that when a light is flashed on on the receptive field, the neuron is excited but a light flashing off evokes an inhibitory response \cite{Palmer1981, Martinez2005, Kremkow2016}. It was suggested that this push-pull interaction of receptive fields involved both Hubel and Wiesel like excitatory convergence as well as intracortical inhibition \cite{Martinez2005, Kremkow2016}. 

\paragraph{2) Spatially offset on and off sub-regions}

In this model, orientation selectivity is generated by the organisation of the on and off sub-fields of the simple cells. According to this model, off inputs to the cortex anchor cortical retinotopy and on afferents organise themselves around the off centres to generate both orientation selectivity and the columnar architecture evident in the cortex \cite{Kremkow2016a} Lee et al 2015. This model was based on the spatial organisation of on and off inputs to the cortex \cite{Jin2011a, Kremkow2016a}, on studies that suggested that developmentally, off inputs arrive first in the cortex, that off thalamic afferents cover lager cortical territory and make stronger connections (Jin et al., 2008; \cite{Jin2011a}).


\paragraph{Evaluation of spatially offset sub-region models}

While there is evidence validating both the spatially offset excitation and inhibition models and the spatially offset on and off sub-region models, these models are unable to explain the effect of abolishing the on inputs to neurons. Schiller and colleagues showed that when APB was applied to simple and complex cells, the neurons no longer responded to light increments but the orientation selectivity of the remaining off response was unaffected \cite{Schiller1992}. If Heggelund's model were true, we could expect that the orientation selectivity of the neuron would remain unaffected but, neurons would still give an "on response" even in the absence of any on inputs from the retina. Such responses have been demonstrated in the on surrounds of off centre neurons in the LGN but not in the cortex. In the push-pull model, we would expect that the orientation selectivity of the off-sub-regions would broaden as the inhibition of the off sub-region from on-centred inhibitory neurons would be removed. Similarly, in models that need the presence of both the on and off sub-regions, the orientation selectivity of the neurons will be broader when the on sub-regions are silenced using APB application in the retina. The APB experiments show that interaction between the on and off excitatory inputs (and excitatory and inhibitory sub-regions) is not necessary for the generation of orientation selectivity in the primary visual cortex. As a result, any models proposing such an interaction need to be able to explain these results within the context of the model.


\subsubsection{Anisotropic LGN driven - recurrent model (ALD-RM)}

The ALD-RM model suggests that orientation selectivity arises in two stages; in the first stage, orientation selectivity is established by biased LGN-inputs. These biases are generated from mild-biases already present in the LGN or but the spatial convergence of inputs that are separated by less than the diameter of a receptive field. There might even be orientation biased inhibitory inputs at this stage. Once this initial selectivity is established, orientation selectivity is sharpened through intracortical mechanisms such as recurrent excitation and cross-orientation inhibition. These mechanisms then institute a spike threshold which causes a tip of the ice-berg effect and the resultant signal is sharply tuned to orientations \cite{Vidyasagar1987, Vidyasagar1996b}.

\subsubsection{Evaluation of the ALD-RM model}

One of the key strengths of the excitatory convergence model is its ability to easily explain the length summation observed in the cortical neurons. Many spatially off-set excitatory regions can intuitively explain the response of a neuron to increasing stimulus length. However, experiments have shown that there isn't a super elongated excitatory regions in neurons (Ref). Such length summation could result from horizontal connections between neurons of the same orientation. Further, inhibitory interneurons that possess end-inhibition (inherited from LGN neurons) could confer length summation on to the target neuron via dis-inhibition.

The ALD-RM model also explains the reduced elongation of receptive fields compared to the LGN receptive fields observed in the cortex. Intracellular studies showed that the aspect ratio of cortical receptive fields was 1.7:1 which was much smaller than the that expected from an excitatory convergent model \cite{Pei1994}. These results would be better explained by the ALD-RM which suggests that the maximum separation between 2 LGN neurons that project on to the same cortical neuron should be less than the diameter of the LGN neuron. This way, the maximum elongation of the receptive field one cam expect is 2:1. 

While the separation between LGN receptive fields projecting to the same cortical neuron maybe restricted, the ALD-RM model is not necessarily contradictory to the cross correlation studies that have shown that upto 10 X-cells converge onto cortical simple cells and as many as 30 LGN Y cells may converge onto a complex cell \cite{Tanaka1983}. However, studies have shown that the pooled inputs from such a sample of LGN neurons could be consistent with inputs arriving from only one or two LGN receptive fields. Further, cross-correlation studies have also shown that striate cortical neurons receive most of their inputs from a small number of retinal receptive fields. The divergence of inputs from the retina to the LGN and the following convergence of inputs from many LGN neurons onto a cortical neuron may not only explain the receptive sizes of these neurons but may also incidentally explain how the small proportion of LGN input to the cortex is "heard".

Intracortical studies have given mixed results on the presence of inhibition in the cortex. As mentioned earlier, early intracellular studies showed the presence of cross-orientation inhibition \cite{Creutzfeldt1974, Pei1994} whereas later studies did not really report any inhibitory response in the cortex \cite{Ferster1986} or tended to report iso-orientation inhibition in the cortex \cite{Anderson2000}. We addressed why we may not see inhibition in the recordings when discussing the recurrent excitation model. As the ALD-R model has a recurrent excitatory component built into it, the model is also capable of addressing this issue. Further studies that examined the inhibitory inputs to the cortex also demonstrated a broadly tuned input, further sharpened by intracortical inhibitory mechanisms \cite{Pei1994, Volgushev1995, Volgushev2000}. The feedforward, orientation non-specific inhibition could also provide initial non-specific shunting inhibion which has been widely reported in the literature (ref).







\section{Receptive field organisation and Spatial Frequency}

Unlike orientation selectivity reviewed earlier, far fewer studies have examined the spatial frequency tuning of neurons. It is acknowledged that the spatial frequency tuning of neurons is an emergent property of the neuron's receptive field organisation- so both these properties are addressed together in this part.
Three aspects of spatial frequency selectivity of the neurons- 1st is the peak spatial frequency of the neuron. 2) Low spatial frequency cut-off and 3) High spatial frequency cut-off of the neuron.


What affects spatial frequency tuning?
Distribution of spatial frequency
Interactions with orientation tuning


\section{Linearity of Spatial Summation}

what does it mean?
classification
Differences

\section{Cortical architecture}

Models that generate
TINS.
\section{Columnar Organisation in the primary visual cortex}
Despite the differences highlighted above, the supragranular layers have similar functional architecture in all three species. Hubel and Wiesel (1962; 1968) first demonstrated the presence of orientation columns in cats and in macaques using electrophysiology. This was also later demonstrated using autoradiographic studies (Hubel et al., 1978). Optical imaging of intrinsic signals showed that orientation in the V1 was organised in columns which converged at pinwheel centres in cats and macaques (Bonhoeffer \& Grinvald, 1991; Bartfeld \& Grinvald, 1992). In the tree shrews, Humphrey and Norton (1980) suggested that orientation columns were organised in elongated columns perpendicular to the V1/V2 border. However, later Bosking et al. (1997) showed using optical imaging of intrinsic signals that orientation columns were organised in a similar fashion to what was observed in macaques and cats. Given this, it may be supposed that while the inputs to V1 in cats, macaques and tree shrew are different, the mechanism through which orientation tuning develops in all three species maybe similar. 

Organisation of orientation selectivity in the cortex.

1) Laminar organisation
2) Columnar organisation
3) Orientation anisotropies

Orientation selectivity in macaques and tree shrews - these are the two species studied here.

