\chapter{Literature Review}

Hubel and Wiesel, in their seminal paper in 1962, described the receptive field (RF) properties and the functional organisation of neurons in the primary visual cortex (V1) of cats. They showed that most cortical neurons were sharply tuned to orientation, showed ocular dominance and could be classified as simple or complex. They proposed a hierarchical, feedforward model that gave rise to orientation selectivity, simple cells as well as complex cells. Further, they proposed that neurons in V1 were grouped into columns based on the optimum orientation of the neurons. Since their first report, the theories proposed by Hubel and Wiesel have been heavily interrogated with evidence presented both to support and oppose their proposals. Some of this literature is reviewed below.

\section{The Visual Pathways}

Information in the visual system flows from the retina to the dorsal lateral geniculate nucleus (dLGN; referred to henceforth as the LGN) to the primary visual cortex (V1) From V1, visual information is further transmitted to extra-striate visual cortices. An alternate pathway to the extra-striate visual cortices also exists. In this section, a short description of the receptive field properties that are studied in this thesis as well as their transformation at each stage of the visual pathway are described.


\subsection{Receptive field Properties}
The receptive field of a neuron "Spatial distribution of responsiveness as mapped literally in the external visual field" (Levick, 1996). 


Hartline (1940) first described the receptive field of optic nerve fibres as the part of the visual space where the activity of the neuron can be influenced \cite{Hartline1940a}. Since their description, the receptive fields of neurons from every visual area have been studied extensively. The receptive field properties that have been studied here are:

\textbf{Polarity preference of neurons}: Starting from the retina to the striate cortical neurons, neurons or sub-regions of neurons respond to light increments (on stimuli) or light decrements (off stimuli). Care must be taken to not interpret these responses as excitation and inhibition respectively. Both on and off stimuli could produce an excitatory response on neurons and can mutually inhibit the other response.

\textbf{Orientation Tuning}: Neurons that selectively respond to edges of certain orientations were first reported in the primary visual cortex in cats \cite{Hubel1962d}. Since its first report, neurons in most regions of the visual pathways have been shown to possess some degree of orientation selectivity.

\textbf{Spatial Frequency Tuning}: Based on the organisation of the on and off subregions, a neuron responds selectively to certain spatial frequencies of the stimulus.

\textbf{Linearity of Spatial Summation}: Within its sub-regions, a linear neuron summates linearly and between the different summation, there is inhibition so that responses of the neuron can be predicted from its receptive field structure.

\subsection{Retina}

The retinal ganglion cells (RGCs) are  the first stage of neural processing in vision. Most of the features observed in the primary visual cortex are already present to some extent in the retinal ganglion cells. In this part, the receptive field properties of the retina are described. Most of these receptive field properties are common to cats, macaques and tree shrews, the three species studied here. Any significant differences between the three species are pointed out.

\paragraph{Receptive Field Structure}

RGCs have concentric centre-surround receptive field structure with neurons either showing a preference for light increments or light decrements \cite{Kuffler1951}. In neurons, the centre responded to light increments while the surround responded to light decrements (on-centre-off-surround neurons) and in off neurons, the centres responded best to light decrements while the surround responded best for light incremements (see fig \ref{fig:rforg}). Later studies demonstrated,  both electrophysiologically and morphologically that RGCs are a diverse group of neurons with neurons. Among these, neurons that have both on-off representation in the receptive field centres (Cleland et al., 1971; Cleland \& Levick, 1974; Stone \& Fukuda, 1974) and those that have three concentric regions --- a centre that was on-off, an immediate surround that was sensitive to off stimuli and third region sensitive to on stimuli--- have been described. However, these neurons form a minority and most neurons show the centre-surround organisation described above.

\begin{figure}[H]
	
	\includegraphics[width=0.5\linewidth]{litrev/retinarf.jpg}
	\centering
	\caption{The different receptive field organisations in the visual system. a) concentric receptive fields- off centre/on surround and on centre/ off surround neurons. receptive fields are slightly elongated. b) cortical simple cells with segregated on and off regions. c) cortical complex cell where on and off stimuli evoke a response at every part of the rf.}
	
	\label{fig:rforg}
\end{figure}

\paragraph{Orientation Tuning}

While it was originally thought the retinal neurons were untuned for orientation, Hammond showed that the RGC receptive fields were often elliptical \cite{Hammond1974}. Following this, Levick and Thibos showed that RGC neurons indeed demonstrated small orientation biases, especially at higher spatial frequencies and that these neurons were predominantly biased for the radial orientation \cite{Levick1980, Levick1982c}. The radial orientation was the orientation of the line joining the centre of the receptive field to the centre of foveal representation. They also found a bias for the tangential orientation (orthogonal to the radial orientation) when alternating gratings were used \cite{Thibos1985}. The orientation biases of RGCs was attributed to the elongated dendritic fields of RGCs \cite{Leventhal1983a}. Later studies showed that the orientation selectivity of RGCs involve complex interactions with an extended surround \cite{Shou2000}.

\paragraph{Spatial Frequency Tuning}

Enroth-Cugell and Robson first suggested that the spatial frequency tuning of RGCs could be modelled by a difference of Gaussian function, with $\sigma$s equal to the radius of the center and the surround of retinal receptive fields \cite{Enroth-cugell1966b}. It has since been shown that RGCs have low pass spatial frequency tuning, that the peak spatial frequency selectivity of RGCs decreases as eccentricity increases and that at higher spatial frequencies, RGCs show orientation selectivity \cite{Levick1980, Levick1982c}. Cat RGCs respond to spatial frequencies from 0 to 1-5 cycles/deg (ref).

\paragraph{Linearity of Spatial Summation}

Enroth-Cugell and Robson (1966) also classified RGCs into X and Y cells based on the linearity of spatial summation they demonstrated within their receptive fields. When shown light spots of increasing sizes, X cells showed responses that indicated linear summation over their receptive fields whereas Y cells did not. A more descriptive classification was suggested by Cleland and colleagues who classified retinal neurons into sustained and transient neurons which corresponded well with the X and Y neurons respectively (Cleland et al., 1971). X and Y cells could both be on and off centred. Y cells usually had larger receptive fields (Linsenmeier et al., 1982). Functionally, X cells have been associated with transmitting colour information while Y cells have been associated with transmitting movement information. W cells, which are said to transmit blue cone information are also present within the retina (Ref) and showed non-linear responses to stimuli. X and Y cells showed 'brisk' response to stimuli while W cells showed 'sluggish' response to stimuli. X cells showed a higher spatial frequency cut-offs when compared to Y cells \cite{Thibos1983}.


A wide variety of RGCs have been described on the basis of morphology and function, with over 22 different types reported (Ref). In the following review, the receptive field structure, the orientation tuning, the spatial frequency tuning and the linearity of spatial summation are described as these are the properties that are most relevant for this thesis. Other prominent features of RGCs include colour and direction selectivity (Ref).

\subsection{The Geniculo-Striate Pathway }

The primary target of retinal neurons is the dorsal lateral geniculate nucleus (dLGN; referred to as LGN throughout this thesis) which then projects to the primary visual cortex (V1) forming the geniculo-striate pathway. While it was thought the the LGN was a relay area, the morphological and electrophysiological properties of its neurons as well as its projections from the visual cortex, brainstem nuclei and perigeniculate nuclei and the presence of inhibitory interneurons within the area has drastically changed this view (see Sherman and Koch, 1981 for review). The receptive field properties of these structures are discussed below.

\subsubsection{Dorsal Lateral Geniculate Nucleus (LGN)}

The LGN receives direct, feedforward input from the retina but also feedback inputs from V1 and the brainstem areas as well as inputs from geniculate inhibitory neurons. While the inhibitory and feedback inputs account for the majority of connections to the geniculate neurons, the feedforward inputs account for the majority of the LGN neurons' receptive field properties. 

\paragraph{Receptive Field Organisation}
As in the retina, the majority of LGN neurons have centre-surround organisation and are either on-centre/ off-surround neurons or off-centre/ on-surround neurons \cite{Hubel1961}. Further, an extra-classical receptive field (ECRF) is also observed in most LGN neurons (ref). While the centre is meant to reflect the feedforward retinal properties, the ECRF is meant to be due to the inhibitory inputs and cortical feedback (ref).


\paragraph{Orientation Tuning}
Similar to RGCs, LGN neurons were first considered to be untuned to orientation. However, it was later shown that LGN neurons were indeed tuned to orientation \cite{Xu2002} and at higher spatial frequencies, this orientation tuning was more prominent \cite{Vidyasagar1982, Vidyasagar1984b, Xu2002, Suematsu2012}. This orientation selectivity has been attributed to a combination of elongated receptive fields and surround suppression \cite{Suematsu2012, Naito2013}.  Studies  in the kitten LGN showed that the neurons were already biased for orientation at the early stages of development, indicating that there may be a morphological basis for orientation tuning in the LGN \cite{Albus1983}. It was suggested that the orientation biases observed in the LGN were a reflection of the orientation biases in the retina \cite{Soodak1987} however, studies have shown that removing the inhibitory input to the LGN neurons reduces the orientation selectivity of these neurons \cite{Vidyasagar1984}. The ECRF of LGN neurons have also been shown to have broad orientation biases which do not always co-incide with the optimum orientation of the CRF \cite{Sun2004,Naito2007}. The majority of the LGN neurons were tuned for the radial orientation \cite{Shou1989, Smith1990b}.

\paragraph{Spatial Frequency Tuning}
Retinal ganglion cells were tuned to higher spatial frequencies than LGN cells (Maffei \& Fiorentini, 1972). Spatial frequency tuning of LGN neurons also tended to reflect the spatial frequency tuning of retinal neurons \cite{So1981}. ECRF tuned to lower spatial frequencies \cite{Sun2004}. Y cells were better tuned at lower spatial frequencies while X cells fired at higher spatial frequencies \cite{Lehmkuhle1980}.

\paragraph{Linearity of Spatial Summation}
On Centre retinal X cells project to on centre LGN X cells- Parallel pathways from retina to LGN to cortex (Cleland et al., 1971). Magno- parvo pathways- similar to the Y/X dichotomy but not needed really. Something about the parallel pathways X cells were tuned to higher spatial frequencies, had higher spatial frequency cut-offs but lower overall firing rates compared to X-cells (Cleland et al., 1971; Derrington \& Fuchs, 1978). X, Y and W cells== Parvo, Magno and Konio? But parvo layers also have Y cells. So not exactly. X cells- smaller receptive field sizes, slower transmission speeds- Forms part of a slow pathway from the retina to the LGN--- then to the cortex? Y cells- lower SF resolution than X cells (Saul \& Humphrey, 1990).

\paragraph{Ocular Dominance}  

In the LGN have to deal with ocular dominance. Arranged in layers within the LGN. There is some mixing? Serve to provide inhibition (Xue et al., 1987)

\subsubsection{The Primary Visual Cortex (V1)}

hubel and wiesel- described receptive fields in the primary visual cortex. Studied a lot. Neurons are orientation selective

X-Y dichotomy= simple/complex dichotomy in cortex? receptive field size and stimulus speed size are similar- Hoffmann et al., 1971.

\paragraph{Receptive field organisation}

Separate on and off sub-regions- simple cells
Mixed on and off sub-regions- complex cells
End-inhibited- hypercomplex cells- proposed an heirarchical model where a simple cell receives from on and off centre LGN neurons, a complex cell receives inputs from many simple cells and a hyper-complex cell receives inputs from many complex cells. 
Complex cell rfs are bigger than simple cells
Majority of simple cells in layer 4 and layer 2/3 cells are in layer 2/3 but also layer 4 has complex cells.

\begin{figure}[H]
	
	\includegraphics[width=0.5\linewidth]{litrev/corticalrf.jpg}
	\centering
	\caption{Examples of simple and complex cells in V1. The simple cells are symmetric on and off centre cell. Anti-symmetric neurons where there are an even number of antagonistic sub-regions can also be found in V1. The complex cell responds well to either light or dark stimuli at any location over its receptive field.}
	
	\label{fig:rforgV1}
\end{figure}

\paragraph{Ocular Dominance}
In V1, neurons show a wide range of ocular dominances. Bell curve. Organised in columns.

\paragraph{Orientation selectivity}

While LGN neurons retain the properties observed in the LGN, layer 4 of the cats is where there is an extensive transformation of receptive field properties for the first time. Unlike lateral geniculate nucleus neurons which respond to wide field, circular flashing stimuli, V1 neurons respond best bars and edges of the optimal orientation \cite{Hubel1962d}. Since then, orientation selectivity has been detected in the primary visual cortex of all species that were studied. In the V1, the distribution of orientation selectivities differ between the macaque, tree shrews and cats. While in the cats, orientation selectivity is observed in the cortical input layer (layer 4), in the macaques and tree shrews, layer 4 neurons show broad orientation biases as described in the LGN. In all three species, supragranular layers are tuned to orientation. Studies that examined the distribution of orientation selectivities also showed that while overall layer 4 neurons showed broader orientation tuning and layer 2/3 neurons showed sharper orientation tuning, individual neurons in each of these layers show a wide range of orientation selectivities in macaques and tree shrews. Simple cells showed sharper orientation tuning than complex cells \cite{Henry1974}

\paragraph{Spatial Frequency Tuning}

\paragraph{Linearity of Spatial Summation}

Other features that are present in neurons- contrast invariance, length tuning and temporal frequency tuning- should mention these- explain the phenomena

\section{Parallel pathways in the primary visual cortex}

In most species, visual information is segregated on the basis of their functional properties into different pathways. In cats, macaques and tree shrews, the functional segregationdiffers. These differences are briefly examined below (also see Figure 1 for summary).

In macaques, chromatic and achromatic information is segregated in different pathwaysin their projections from retina to LGN to V1. The magnocellular pathway (M-) transmits achromatic information and the neurons in this pathway respond to luminance changes (Hicks et al., 1983; Kaplan et al., 1990; Dacey, 2001). The parvocellular (P-) (Hicks et al., 1983; Kaplan et al., 1990; Merigan \& Maunsell, 1993) andkoniocellular (K-) (Dacey, 2001; Roy et al., 2009)pathways transmit chromatic information. The  major targets of these projections in macaques are in layer 4Cα, 4Cβand layer 3B of V1 (Casagrande \& Kaas, 1994). This segregation was believed to be maintained even in extrastriate areas (Bullier \& Henry, 1980; Casagrande \& Kaas, 1994). However, there is evidence to suggest that there is considerable overlap in inputs as early as layer 4 (Casagrande \& Kaas, 1994; Callayway, 1998; Vidyasagar et al., 2002).

In comparison, LGN inputs to V1 in the tree shrew are segregated into ON, OFF and W-cell pathways (Conway \& Schiller, 1983; Conley et al., 1984; Holdefer \& Norton, 1995). ON cells respond to increases in luminance and OFF cells respond to decreases in luminance. The ON, OFF and W-cells terminate in layers 4A, 4B and 3C of V1 respectively (Conley et al., 1984). Layer 4A mostly have on neurons and 4B, mostly off neurons (for review, see Fitzpatrick, 1996).

In cats, the inputs to V1 segregate differently. X and Y cells of the LGN project to layers 4C and 4A+B respectively (Wilson et al., 1976; LeVay \& Gilbert, 1976). X-cells show a sustained response when presented a stimulus. They also sum signals linearly within their receptive fields. That is, when presented with dark and light stimulus regions over the receptive field at the appropriate phase, there is virtually no response as the cell sums the signals from the ON and OFF sub-regionslinearly. Y cells on the other hand sum non-linearly within their receptive fields and they also have a transient response when a stimulus is presented, irrespective of phase (Enroth-Cugell \& Robson., 1966).
While there are major differences in physiological properties of the different pathways, some similarities have been found. For example, it has been shown that there is some extent of on/off segregation as observed in the tree shrew within the parvocellular layers of the macaque LGN (Schiller \& Malpeli, 1978). It was also originally thought that parvocellular cells were X-cells and magnocellular cells were Y-cells (Dreher et al., 1976). However, this is not entirely the case. While most P-cells are indeed X cells, 75\% of M- cells are also X-cells in macaques (Shapley et al., 1981). Similarly, most neurons in the tree shrew LGN are also X cells, with cells showing non-linear summation only observed in 2 of the 6 layers (Conway \& Schiller, 1983).

\section{Columnar Organisation in the primary visual cortex}
Despite the differences highlighted above, the supragranular layers have similar functional architecture in all three species. Hubel and Wiesel (1962; 1968) first demonstrated the presence of orientation columns in cats and in macaques using electrophysiology. This was also later demonstrated using autoradiographic studies (Hubel et al., 1978). Optical imaging of intrinsic signals showed that orientation in the V1 was organised in columns which converged at pinwheel centres in cats and macaques (Bonhoeffer \& Grinvald, 1991; Bartfeld \& Grinvald, 1992). In the tree shrews, Humphrey and Norton (1980) suggested that orientation columns were organised in elongated columns perpendicular to the V1/V2 border. However, later Bosking et al. (1997) showed using optical imaging of intrinsic signals that orientation columns were organised in a similar fashion to what was observed in macaques and cats. Given this, it may be supposed that while the inputs to V1 in cats, macaques and tree shrew are different, the mechanism through which orientation tuning develops in all three species maybe similar. 


\section{Mechanisms underlying feature selectivity}

Of all the findings reported by Hubel and Wiesel, perhaps the most striking feature of cortical cells is the orientation selectivity of the neurons. While recording from the primary visual cortex, H\&W reported that the neurons preferred edges. Further, they found that the neuron only preferred edges of a very specific orientation often not responding at all to orientations 90$^o$ apart. This was in stark contrast to RFs found in the retina and lateral geniculate nucleus (LGN) neurons that showed poor orientation selectivity. H\&W, in order to explain this sudden rise of orientation selectivity, proposed the excitatory convergence model of orientation selectvity.

\subsection{Models of Orientation selectivty}

\subsubsection{Excitatory convergence}

\paragraph{The Model}Hubel and Wiesel suggested that the orientation selectivity of the cortical neurons came from the excitatory convergence of inputs from circular LGN receptive fields which are arranged in a row. The cortical neuron would then be tuned to the same orientation as the line joining the centres of LGN receptive fields. When the LGN neurons encounter an edge of the optimum orientation, then the centers of all the LGN neurons are simultaneously activated and hence the cortical neuron gives a strong response. When an edge of the non-optimum orientation is presented however, the LGN neurons fire sequentially and as a result, the cortical neuron gives a weaker output (Figure \ref{fig:HW}). This model not only explained orientation selectivity but also the fact that cortical neurons preferred longer bars when compared to LGN neurons and that the LGN and the cortical RF sub-regions had similar widths.

Hubel and Wiesel also described the simple cell. One key feature of the simple cells was that they had spatially segregated responses to light increments (on stimuli) and light decrements (off stimuli). They proposed that the on sub-regions of the simple cell receptive fields received inputs from on-centred LGN RFs and the off sub-regions received inputs from off-centre LGN RFs. Inputs to the different sub-regions of the simple cells arise from LGN neurons of the same polarity (Tanaka, 1983; Chapman et al., 1991; Reid \& Alonso, 1995; Mooser et al., 2004; Jin et al., 2008, 2011). 

	\begin{figure}[H]
	
	\includegraphics[width=\linewidth]{litrev/HW1962.jpg}
	\caption{The model of excitatory convergence proposed by Hubel and Wiesel (1962). LGN neurons with unoriened receptive fields project to Layer 4 neurons in the primary visual cortex, which is tuned to the orientation collinear to the organisation of the LGN neurons receptive field centres.}
	\label{fig:HW}
	\end{figure}

Evidence for Hubel and Wiesel's model
1) The orientation selectivity of LGN inputs to the primary visual cortex are already sharply tuned to orientation when studied using intracellular recordings (Ferster, 1986; Ferster, 1996); when the cortical circuits were silenced using cooling (Ferster et al., 1996); using pharmacological interventions (Nelson et al., 1994; Chapman et al., 1991); using electrical stimulation (Chung \& Ferster, 1998; Kara et al., 2002).
2) LGN inputs to the different sub-regions of simple cells are arranged as Hubel \& Wiesel predicted when studied using cross-correlation studies (Tanaka, 1983; Reid \& Alonso, 1995); morphological studies (Mooser et al., 2004); multielectrode, simultaneous cross-correlation study (Jin et al., 2011).
3) Presence of many simple cells in the cortical input layers (Crowder et al., 2007; Gilbert, 1977, Gilbert \& Wiesel, 1979; Hirsch et al., 1998a; Hirsch et al., 1998b; Kelly \& Van Essen, 1974; Martinez et al., 2002; Martinez et al., 2005; Ringach et al., 2002).
4) Length summation and end inhibition Gilbert, 1977; Rose, 1977; Henry et al.,1978).

Another explanation

1) Orientation selectivity is sharply tuned to orientation - Pei et al., 1994? Showed that input was broadly tuned to orientation. Cortical cooling experiments- not cool cortex enough? Kara et al, vs. Viswanathan et al.
2) LGN inputs to different simple cells- read the original papers.
3) Layer wise organisation of simple and complex cells- not really- first order complex cells.
4) disinhibition.

Issues of the excitatory convergence model and key opponents.

Adjusted excitatory convergence model
Priebe and Ferster- feedforward + non-linearities.

\subsubsection{The Role of inhibition}

\cite{Bonds1989} Cats Used two super-imposed stimuli and found that both orientation non-specific inhibition and orientation specific inhibition played a role in inhibiting responses of neurons.

\cite{Malach1993} Macaques Optical Imaging and biocytin injections- Found connections between same eye neurons. Binocular neurons projected to binocular neurons. Orientation not as fidel. Less projections to orthogonal orientaions.

\cite{Maldonado1997} Cats Cells in pinwheel centres have a similar degree of orientation selectivity as the neurons in isoorientaiton domains. Pinwheels are where all the different orientation domains converge.

\cite{Bartfeld1992a} Macaques Pinwheels at the border between ocular dominance columns. CO blobs also on the border between OD columns but did not co-incide with the pinwheels. Cortical hypercolumn consisting of two OD columns and two pinwheel centres. 

\cite{Gardner1999} Cats Non-linear mechanisms help sharpen orientation selectivity of neurons + make the orientation of neurons independent of the orientation of its sub-divisions.

Alternate models

1) Inhibition; Recurrent excitation
explain model- support and drawbacks

2) Relative arrangement of the on and sub-regions
explain models- support and drawbacks.

3) Oriented inputs
explain model- support and drawbacks.


Organisation of orientation selectivity in the cortex.

1) Laminar organisation
2) Columnar organisation
3) Orientation anisotropies

Orientation selectivity in macaques and tree shrews - these are the two species studied here.


\section{Spatial Frequency}

What affects spatial frequency tuning?
Distribution of spatial frequency
Interactions with orientation tuning
Spatial frequency selectivity in macaques and tree shrews

\section{Linearity of Spatial Summation}

what does it mean?
classification
Differences

\section{Cortical architecture}

Models that generate
TINS.


Most geniculate neurons are excited by neurons of the same type- mixes only with respect to sustained/transient type (Cleland et al., 1971)

Visual pathway of sustained/ transient remains separate from retina all the way to the cortex (Cleland et al., 1971).

