\chapter{Literature Review}

Hubel and Wiesel, in their seminal paper in 1962, described the receptive field (RF) properties and the functional organisation of neurons in the primary visual cortex (V1) of cats. They showed that most cortical neurons were sharply tuned to orientation, showed ocular dominance and could be classified as simple or complex. They proposed a hierarchical, feedforward model that gave rise to orientation selectivity, simple cells as well as complex cells. Further, they proposed that neurons in V1 were grouped into columns based on the optimum orientation of the neurons. Since their first report, the theories proposed by Hubel and Wiesel have been heavily interrogated with evidence presented both to support and oppose their proposals. Some of this literature is reviewed below.

\section{Orientation selectivity}

Of all the findings reported by Hubel and Wiesel, perhaps the most striking feature of cortical cells is the orientation selectivity of the neurons. While recording from the primary visual cortex, H\&W reported that the neurons preferred edges. Further, they found that the neuron only preferred edges of a very specific orientation often not responding at all to orientations 90$^o$ apart. This was in stark contrast to RFs found in the retina and lateral geniculate nucleus (LGN) neurons that showed poor orientation selectivity. H\&W, in order to explain this sudden rise of orientation selectivity, proposed the excitatory convergence model of orientation selectvity.

\subsection{Models of Orientation selectivty}

\subsubsection{Excitatory convergence}

\paragraph{The Model}Hubel and Wiesel suggested that the orientation selectivity of the cortical neurons came from the excitatory convergence of inputs from circular LGN receptive fields which are arranged in a row. The cortical neuron would then be tuned to the same orientation as the line joining the centres of LGN receptive fields. When the LGN neurons encounter an edge of the optimum orientation, then the centers of all the LGN neurons are simultaneously activated and hence the cortical neuron gives a strong response. When an edge of the non-optimum orientation is presented however, the LGN neurons fire sequentially and as a result, the cortical neuron gives a weaker output (Figure \ref{fig:HW}). This model not only explained orientation selectivity but also the fact that cortical neurons preferred longer bars when compared to LGN neurons and that the LGN and the cortical RF sub-regions had similar widths.

Hubel and Wiesel also described the simple cell. One key feature of the simple cells was that they had spatially segregated responses to light increments (on stimuli) and light decrements (off stimuli). They proposed that the on sub-regions of the simple cell receptive fields received inputs from on-centred LGN RFs and the off sub-regions received inputs from off-centre LGN RFs. Inputs to the different sub-regions of the simple cells arise from LGN neurons of the same polarity (Tanaka, 1983; Chapman et al., 1991; Reid \& Alonso, 1995; Mooser et al., 2004; Jin et al., 2008, 2011). 

	\begin{figure}[H]
	
	\includegraphics[width=\linewidth]{litrev/HW1962.jpg}
	\caption{The model of excitatory convergence proposed by Hubel and Wiesel (1962). LGN neurons with unoriened receptive fields project to Layer 4 neurons in the primary visual cortex, which is tuned to the orientation collinear to the organisation of the LGN neurons receptive field centres.}
	\label{fig:HW}
	\end{figure}

Evidence for Hubel and Wiesel's model
1) The orientation selectivity of LGN inputs to the primary visual cortex are already sharply tuned to orientation when studied using intracellular recordings (Ferster, 1986; Ferster, 1996); when the cortical circuits were silenced using cooling (Ferster et al., 1996); using pharmacological interventions (Nelson et al., 1994; Chapman et al., 1991); using electrical stimulation (Chung \& Ferster, 1998; Kara et al., 2002).
2) LGN inputs to the different sub-regions of simple cells are arranged as Hubel \& Wiesel predicted when studied using cross-correlation studies (Tanaka, 1983; Reid \& Alonso, 1995); morphological studies (Mooser et al., 2004); multielectrode, simultaneous cross-correlation study (Jin et al., 2011).
3) Presence of many simple cells in the cortical input layers (Crowder et al., 2007; Gilbert, 1977, Gilbert \& Wiesel, 1979; Hirsch et al., 1998a; Hirsch et al., 1998b; Kelly \& Van Essen, 1974; Martinez et al., 2002; Martinez et al., 2005; Ringach et al., 2002).
4) Length summation and end inhibition Gilbert, 1977; Rose, 1977; Henry et al.,1978).

Another explanation

1) Orientation selectivity is sharply tuned to orientation - Pei et al., 1994? Showed that input was broadly tuned to orientation. Cortical cooling experiments- not cool cortex enough? Kara et al, vs. Viswanathan et al.
2) LGN inputs to different simple cells- read the original papers.
3) Layer wise organisation of simple and complex cells- not really- first order complex cells.
4) disinhibition.

Issues of the excitatory convergence model and key opponents.

Adjusted excitatory convergence model
Priebe and Ferster- feedforward + non-linearities.

Alternate models

1) Inhibition; Recurrent excitation
explain model- support and drawbacks

2) Relative arrangement of the on and sub-regions
explain models- support and drawbacks.

3) Oriented inputs
explain model- support and drawbacks.


Organisation of orientation selectivity in the cortex.

1) Laminar organisation
2) Columnar organisation
3) Orientation anisotropies

Orientation selectivity in macaques and tree shrews - these are the two species studied here.


\section{Spatial Frequency}

What affects spatial frequency tuning?
Distribution of spatial frequency
Interactions with orientation tuning
Spatial frequency selectivity in macaques and tree shrews

\section{Linearity of Spatial Summation}

what does it mean?
classification
Differences

\section{Cortical architecture}

Models that generate
TINS.