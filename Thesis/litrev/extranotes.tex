\chapter{Extra notes}
Most geniculate neurons are excited by neurons of the same type- mixes only with respect to sustained/transient type (Cleland et al., 1971)

Visual pathway of sustained/ transient remains separate from retina all the way to the cortex (Cleland et al., 1971).

\cite{Malach1993} Macaques Optical Imaging and biocytin injections- Found connections between same eye neurons. Binocular neurons projected to binocular neurons. Orientation not as fidel. Less projections to orthogonal orientaions.

\cite{Maldonado1997} Cats Cells in pinwheel centres have a similar degree of orientation selectivity as the neurons in isoorientaiton domains. Pinwheels are where all the different orientation domains converge.

\cite{Hubel1969} macaques Electrode penetations- orientation and ocular dominance columns.s
\cite{Bartfeld1992a} Macaques Pinwheels at the border between ocular dominance columns. CO blobs also on the border between OD columns but did not co-incide with the pinwheels. Cortical hypercolumn consisting of two OD columns and two pinwheel centres. 

They also showed that the cortical RF structure is not elongated when using spots of light but rather circular. The bar responses using bars is actually elongated because the bar might activate some of the inhibitory regions. Does not say inhibition is cross orientation- rather that inhibition comes from the same orientaiton. 

Intracellular inhibition tuned to the same orientation as the neurons orientation \cite{Anderson2000}.

Does not really explain orientation selectivity- could at best sharpen orientation selectivity.

\cite{Priebe2006} Cats, Threshold can explain the effects of cross-orientation suppression. Don't really explain the effects of Eysel et al.

\cite{Kisvarday1997} Cats, Traced the horizontal connections- majority of excitatory and inhibitory connections to iso-orientation domains (~40\%). More in support of an overall non-specific inhibition from pooled responses.

\cite{Morrone1982}:Cats, Used Noise of optimal orientation and stimuli of orthogonal orientation- Found cross orientation inhibition- better for simple cells than complex cells. Inhibition in most orientations, facilitation in the same orientation- inhibition pooled from many cortical neurons? Spatial frequency also pooled across many different SFs.

\cite{Worgotter1988, Eysel1990} Cats Area 17: Recorded from a neuron while suppressing activity in neurons located ~ 500 microns away from the recorded neuron- found that orientation tuning broadened in the recorded neuron. Evidence for inhibition from neurons of other orientations due to the periodicity of cortical orientation tuning. Broadening of orientation selectivity due to an increase in firing rate at the orthogonal orientation. 

\cite{Crook1992}: As above but in Area 18. More inhibition from neurons that are tuned 45 degrees apart than less than 22.5 degrees apart. When iso-orientaiton inhibition- causes increase in firing rate more than anything else.

\cite{Sharon2002}: With Grinvald- showed that orientation of the neurons are potentially inherited from the feedforward signal but the amplification involves both intracortical excitation and inhibition in cats, VSDs.

\cite{Somers1995} Computational modelling- neuron received weakly tuned thalamic signal, strong iso-orientation inhibition, weak cross orientation inhibition and no shunting inhibition. Found that neurons were sharply tuned to orientation. Suggest that inhibition occurs in an orientation non-specific manner. Orientation tuning is an emergent property of the cortex.

Another explanation

1) Orientation selectivity is sharply tuned to orientation - Pei et al., 1994? Showed that input was broadly tuned to orientation. Cortical cooling experiments- not cool cortex enough? Kara et al, vs. Viswanathan et al.
2) LGN inputs to different simple cells- read the original papers.
3) Layer wise organisation of simple and complex cells- not really- first order complex cells.
4) disinhibition.

Issues of the excitatory convergence model and key opponents.

Adjusted excitatory convergence model
Priebe and Ferster- feedforward + non-linearities.


\textbf{Recurrent models}

\cite{Somers1995, Douglas1995}. Further, it was suggested that instead of just subtractive inhibition, if shunting inhibition was employed, the network was more stable (Heeger 1992, 1993, Carandini \& Heeger 1994, Chance \& Abbott 2000).

\cite{Adorjan1999} No need for biases in the feedforward inputs- can be generated completely from intracortical mechanisms.

\cite{Carandini1997} Modelled Somers and Douglas and found that with recurrent excitation alone, the model was really bad at distinguishing between orientations. Requires sharply tuned inhibition. Non-specific inhibition doesn't really work. This is opposite of what what has been reported. 


1) anatomy: neurons receive mostly intracortical, excitatory connections. 
2) Dynamics: Dynamics of orientation selectivity show that neurons get more and more tuned to orientation as time progresses- Pei et al., 1994; Ringach 2002. 
3) Intracortical inhibition- inhibition tuned to the same orientation as the excitation.

\cite{Nelson1998} Galaretta and Hestrin show that prolonged firing causes a much bigger depression at excitatory synapse rather than at inhibitory synapses.

Early evidence for the recurrent excitation model comes from 
\cite{Shu2003}: Invitro slices- showed that excitation and inhibition were balanced. 

\cite{Monier2003}: Most excitatory and inhibitory connections are tuned to the same orientation as the recorded neuron.

\cite{Li2013}: Intracortical excitatory circuits linearly amplify the thalamocortical information and may influence the size of the receptive field by recruiting additional inputs

\cite{Sato2016} Neural networks give shunting inhibition by moderating the level of excitation than by increasing inhibition.

\cite{Tan2011a} mouse and cats- inhibition is the same orientation as excitation. Sharpening of orientation selectivity by changes in input conductances.

\cite{Shapley2007} INtracortical inhibition is important for orientations selectivity.
\section{Orientation Selectivity}
\cite{DeValois1982}: Macaques neurons in V1 are tuned to orientation and direction. Different degrees of orientation selectivity within simple/complex cells. Within foveal neurons, vertical and horizontal orientations are better represented. Most inhibition at orientations adjacent to the orthogonal orientation rather than at the orthogonal orientation- models.

\cite{Fregnac1978} kittens- sharp orientation selectivity from 12 days already but ocular dominance takes some time. Also binocular inputs required to generate the full range of orientations preferences in kittens. kittens under 3 weeks of age a higher proportion of binocular neurons. Horizontal and vertical biases in early stages of development.

\cite{Sherk1975} Also kittens- sutured eyes= recorded at 22-23 days- already tuned to orientation. 

\cite{Cynader1975} Cats reared in a room with unidirectionally moving stimuli- direction selectivity- direction of the moving grating (leftward). Most units were oriented vertically. No such effect in the SC. 

\section{Spatial Frequency Tuning}

\cite{Skottun1987} Contrast invariant spatial frequency tuning. 

\cite{DeValois1982a} Described the old modulation ratio. Striate cells have quite narrow band pass spatial frequency bandwidths, at a given retinal eccentricity , the distribution of peak frequency covers a wide range of frequencies.  Foveal samples extend into higher spatial frequencies as do complex cell- this last bit doesn't make sense as generally complex cells are bigger and so have lower spatial frequencies.

\cite{Vidyasagar1994a} Showed that neurons in the primary visual cortex started responding at lower spatial frequencies when Bicuculline was applied. orientation non-specific inhibition to neurons.

\cite{Meese2004} Cross orientation inhibition is more visible at lower spatial frequencies.

\cite{Bredfeldt2002a} Suppression at lower spatial frequencies. Temporally delayed. Talks about the two component model?

\cite{Bauman1991} Orientation non-specific, spatial frequency specific inhibition.

\cite{Morrone1982} Inhibition is broadly tuned for spatial frequency.

\cite{DeAngelis1992} Suppresion is broadly tuned for spatial frequnecy. 

\cite{David2004} When using natural images, the spatial frequency tuning of the inhibition undergoes complex changes.

\cite{Field1986} Simple cell receptive fields could be modelled by Gabor functions that had 3 free parameters- Envelope width, carrier width and carrier frequency. 
	
	\cite{DeValois1982a} Described the old modulation ratio. Striate cells have quite narrow band pass spatial frequency bandwidths, at a given retinal eccentricity , the distribution of peak frequency covers a wide range of frequencies.  Foveal samples extend into higher spatial frequencies as do complex cell- this last bit doesn't make sense as generally complex cells are bigger and so have lower spatial frequencies (Fig6).
	Receptive field size of complex cells only have a very small relationship with the peak spatial frequency- not a direct X/Y relationship.
	In simple cells the relationship is more obvious. 
	
	\cite{Silverman1989} Found a continuum of spatial frequencies rather than two distinct groups- high spatial and low spatial frequencies. Lower spatial frequency neurons were found in cytochrome oxidase blobs.
	