Literature Review Notes

Retina
Classification into X and Y cells- Enroth -Cugell \& Robson (1966)

classified into sustained and transient cells- Cleland et al., 1971

Sustained RGCs had slower conducting axons (Cleland et al., 1971)
X-cells- show fairly linear spatial summation over receptive field.

Y-cells- show fairly non-linear spatial summation Enroth-Cugell \& Robson, 1966).

Both on and off centre cells could be subdivided into X and Y cells (ECR, 1966)

Sustained (X like cells)- Continued to respond till the stimulus was presented to the receptive field
Transient (Y like cells)- Responded well when stimulus was presented and then within seconds the response dropped down to spontaneous levels within seconds.

X cells in LGN- X- single input and X-lagged- two different types of responses.

Linsenmeieretal., 1982 

The contrast sensitivity to gratings drifting at 2.0 Hz has been measured for X and Y type retinal ganglion cells, and these data have been used to characterize the sizes and peak sensitivities of centers and surrounds. The assumption of Gaussian sensitivity distributions is adequate for both types of cells, but allows a better description of X than of Y cells. The size and peak sensitivity can be specified more precisely, in general, for the center than for the surround. The data also show that for both types of cells 1. (1) center radius increases with eccentricity, but is two to three times larger for Y cells than for X cells at a given eccentricity 2. (2) spatial resolution is an excellent predictor of center size 3. (3) the larger the center or surround, the lower its small spot sensitivity at a specific mean luminance 4. (4) the surround is nearly as strong as the center for large or diffuse stimuli. X cell surrounds are relatively weaker in the middle of the receptive field than Y cell surrounds, but X cell surrounds are larger relative to their centers.

W-cells- Tonic (on centre and off centre) and phasic(on centre/off centre/on-off centre)

Shou et al., 2000: Found that extended surrounds of the retinal ganglion cells showed orientation selectivity even if centre wasn't tuned to orientation. The centre+surround, center and surround showed orientation tuning at different spatial frequencies.- neurons in the retina can process more complex stimuli.

Based on the size of the receptive fields and morphology of the neurons, there are meant to be over 20 different types of RGCs. For the purpose of this thesis, we mainly discuss the receptive field properties of the alpha and beta RGCs (DOUBLE CHECK THAT THIS IS A THING).

Centre-surround organisation in the retina (Kuffler, 1952, 1953); On and off centre neurons

"This unit had an on-off receptive field and behaved in some respects like the local-edge-detector ganglion cells in the visual streak of the rabbit's retina (Levick, 1967). Cleland et al., 1971"