\chapter{Literature Review}

\section{Visual Pathways}

\subsection{Superior Colliculus}


\subsubsection{Functional Organisation of the Superior colliculus}

The superior colliculus has two major functional subdivisions. The superficial layers are primarily associated with form perception while the intermediate and deeper layers are associated with eye movements (especially generating saccades)and orienting behaviour. Studies where the superficial layers of the superior colliculus were lesioned, researchers found that animals lost the ability to discriminate between stimuli whereas lesioning the deeper layers also significantly impaired animals' ability to orient to a stimulus (cats; macaques; shrews; Casagrande, 1972). Though both the superficial layers and the deep layers of the superior colliculus perform important functions, as this thesis only focuses on the visual pathways, the organisation of the superficial layers of the SC and their receptive field properties are discussed in further detail.

The superficial SC is further subdivded into three sub-layers. The outermost of these layers is the stratum zonale. This layer consists mostly of fibres. Just below the SZ is the stratum griseum superficiale (SGS). In macaques and tree shrews, the SGS is further differentiated into upper (uSGS) and lower(lSGS). In cats, uSGS has layers 1 and 2. Just below the SGS is the final layer of the SGS, the stratum opticum or SO. The SZ, SGS and SO together form the superficial Superior Colliculus and projections to and from the visual regions of the brain (retinal, cortical and brainstem regions) terminate in this regions \cite{Swisher2010}.
\subsubsection{Connections of the Superior Colliculus}


\subsubsection{Receptive Field Properties of the superficial Superior Colliculus}

The most salient properties of the receptive fields of superior layers of the superior colliculus reported in the literature are binocularity and direction selectivity. Due to an almost complete decussation of retinal inputs to the superior colliculus, most SGS neurons receive inputs from the contralateral eye only. As a result, in most species, the binocularity of the cells has been attributed to cortical feedback. Similarly, the direction selectivity of superior colliculus has also been attributed to cortical feedback, although the proportion of direction selective cells reported in individual species are different. For example, nearly 75\% of neurons recorded in the cats were direction selective whereas the proportion was on 10\% in the macaque superior colliculus. In the tree shrew superior colliculus almost no cells are reported as direction selective, reflecting the lack of direction selectivity reported in the tree shrew cortex, further supporting the theory that direction selectivity reported in the SC are derived from the cortical neurons. Lesion studies have also reported that ablating the primary visual cortex in cats abolishes both binocularity and direction selectivity in the superficial layers of the superior colliculus.

Most superior colliculus neurons receive inputs from Y and Z cells. In cats and macaques, the superior colliculus neurons are not tuned to orientation. There are a few neurons in the tree shrew that are said to be tuned to orientation. They have large receptive fields.

\subsection{Orientation Anisotropies in the primary visual cortex}

Sasaki et al 2006
'A link between orientation selectivity and cortical retinotopy which has previously been considered independent'
Higher contrast sensitivity for the radial orientation bias in the periphery
A more global radial orientation bias- based on the quadrants
Psychophysics-they show that there is an interaction between orientation and location. There is still as higher activation for the horizontal and vertical orientations but also because the are radial?

They did test in a little bit of a greater detail by testing orientation bias near the horizontal and vertical meridians. But we have exact retinotopic location for receptive fields - Account for jitter?
Also tested for oblique locations- Then found activation in strips

Retinotopically limited stimulus

Differences: We have better spatial resolution. OI, Filtered

	Something about visuotopical organisation- being able to generate local properties from global organisation properties See Alexander et al 2004
	Ability to pool across long range receptive field connections?
	
	local organisation from global organisation of non- retinotopic settings- no not really. This was pre on-off stuff
	
Ringach 2007

Cat V1 Neurons are mostly contra dominated and Off centred.
Talks about Crair et al., 1998 who suggests that there is contralateral input which sets up the orientation columns and the ipsi input goes along for the ride- Idea similar to Alonso who also says that there is off input and the on goes along for the ride.

One important critique of sagar's model is then

What if orientation columns were not really a thing? There is adaptation- neurons change their orientations. What if this plastic process was ongoing? Just by measuring it, are we changing what is there?



Difference between the time course and spatial resolution.

Difference between the relationship between LFPs and multi unit activity: There isn't a relationship.



\subsection{Spatial Frequency Tuning}

\bibliographystyle{apacite}

\bibliography{Bibtex/library}

