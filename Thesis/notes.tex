	\chapter {Notes- Tree Shrew}
	
	\textbf{Gardner et al., 1999}
	Simple cells have non-linear and linear components: This assumption is by definition untrue: Simple cells are meant to be predominantly linear.
	But the premise of the paper is that because inputs to the simple cells are largely linear, any non-linearity is cortical. This is true but then we will have to look at thresholding. What is the underlying mechanism for this? Is this considered cortical? Recurrent excitation raises the spike threshold?
	
	Marked RF using binary m-sequency noise- What is this? This is the white noise method.
	Suggest that cortical mechanisms make ori tuning less dependent on the size and shape of the receptive field itself. Once again, this may not be true.
	
	Start working on the assumption that inputs are circular.
	
	\textbf{Ferster et al., 1996}
	
	Cooling the cortex largely reduced the cortical activity but geniculate input was still funtional.Hmm they used optimum SF for the grating and still observed that orientation tuning on V1. But then there is still significant projections from area 18. What does this mean really? 
	Seems to suggest that geniculate input is already strongly oriented. But how does area 18 affect this is the question?
	
	\textbf{Daugman 1979}: Pretty much berates the world for looking at spatial things in one dimension. In particular suggests that if doing a one dimensional analysis, essentially the orientation tuning will change at different spatial frequencies, unless the underlying receptive field is circularly symmetrical.
	Suggests that we need to look at the two dimensional spatial profile and the predictions can then actually be compared to the orientation and spatial frequency profiles of V1 to actually quantify responses.
	
	\textbf{Chung and Ferster, 1998}
	
	Intracortical inhibition was strongest at the preferred orientation rather than the orthogonal orientation. Need to re-read. How independent can you really be from your own results.
	
	Question for radial bias: Sagar- in Errol's paper, we say that individual fibres in the orientation columns are tuned to the orientation of the column. But here we are saying that they are all tuned to the radial. Aren't these two ideas incongruent with each other? Am I missing something?
	
	Evidence that GABA is the only inhibitory neurotransmitter in the visual system?
	
	\textbf{Chisum et al, 2003}
	
	Main conclusions: Orientation tuning in layer 2/3 is sharper than that observed in layer 4. MDF: more elongated in layer 2/3 compared to layer 4. Do they actually say that the MDF is 6 degrees long for layer 4 neurons? also 9 degrees long for layer 2/3neurons. What does the MDF actually correspond with? Is this what we record?
	Length summation in layer 2/3 cells. But are they tuned to length?
	Length suppression in layer  4- which is what you observe in LGN of cats. But also lack of length suppression in some layer 4 cells... What does that mean? Do all neurons in LGN of cats show length suppression?
	If layer 2/3 showed length summation: then it would make sense that putting the gabors of optimum orientation in a collinear fashion will facilitate.
	Keeping the center the same and rotating surrounding Gabors: not significantly greater than presenting individual Gabor: Which is kind of the same as the end inhibition experiment which suggests that excitation is only really in the middle area... right?
	Essentially says that eventhough they think that horizontal connections are important, at the end of the day, it's the layer 4 inputs that matter. Also that layer 4 neurons that are present far away can affect activity.
	
	\textbf{Comment: Fitzpatrick 2000}
	Mentions that the length summation area changes according to contrast.
		Contrast is another crazy dimension that I haven't even thought about.
		At lower contrast the visual system may compromise resolution for detection?
		Also changes according to attention. Hmmm....
		
	\textbf{Conley, Fitzpatrick and Diamond, 1983}
	 

	 \textbf{Schiller and Malpeli 1978}
	 Differences between magno and parvocellular cells in the macaque LGN: 
	 Parvo= slow, sustained responses, some are color tuned. Others show broadband: Layers 5 and 6 are predominantly on and layer 3 and 4 are predominantly off. Blue on cells predominantly in layer 3 and 4.
	 Magno= fast, transient response, broadband
	
	 \textbf{Malpeli and Schiller 1978}
	 Blue cones are predominantly on centred. Off centred blue cones are rare. Surround of blue on cells either red or green or both. Same response as that of on centre. Whereas surround of red or green on
	centre cells, response is a lot smaller. Similar trend in LGN.
	Could then indicate different ways in which colour opponency is created. Then is there already a precedent for single and double row projections in the retina.
	
	Orientation columns are a key organising feature of the visual cortex
	
	
	 \textbf{Random Thoughts}
	 Does the nature of neuroscience mean that there is no way to quantify it meaningfully???? What do the numbers really mean?
	 	Orientation columns are a key organising feature of the visual cortex