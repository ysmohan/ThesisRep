\chapter{Relationship between orientation tuning and spatial frequency tuning in the tree shrew V1}
\pagebreak
\section{Summary}
\pagebreak
\section{Introduction}

Early studies conducted in the tree shrew primary visual cortex indicated that orientation tuning in the V1 of tree shrews may be generated from excitatory convergence of unoriented, layer 4 neurons onto layer 2/3 neurons. However, the authors acknowledged that while this excitatory convergence is capable of providing orientation biases, the extensive horizontal connections present in the superficial layers of the tree shrew play an important role in sharpening these orientation biases (Chisum et al., 2003; Mooser et al., 2004). Later however, it was shown that layer 4 neurons did not have circular receptive fields as was originally thought but had broad orientation biases (Van Hooser et al., 2013). A separate study by Veit et al (2014) also argued that horizontal connections in tree shrews are important as just the orientation tuning of inputs seemed insufficient to predict the degree of orientation selectivity of the layer 2/3 cells.
Huang et al (2014) when they tried to test how the horizontal connections worked, didn’t really find what they had hoped. Found that horizontal connections contributed linearly to cell responses regardless of the orientations of where the horizontal connection terminated. They also did not find any axial effects as has been predicted in the past. Issues- they  could just be stimulating within 500 microns, where horizontal connections are not specific? Recurrent excitation? Also they could selectively activate only excitatory neurons using their viral vectors which could leave and inhibitory modulatory circuits out.
Recently, using two photon calcium imaging, Lee et al, 2016, suggested that off inputs to the cortex are established by on inputs organising themselves around off inputs which establish topography. However, there are a few caveats to this model. Muly and Fitzpatrick (1992) showed that on and off inputs to layer 2/3 cells have significant overlap. Further, Veit et al (2014) showed that only 7\% of all cells in the shrew V1 had segregated receptive field sub-divisions, lacking the basic RF structure for the majority of the cells to develop orientation selectivity using this method. 



\section{Methods}


\subsection{Surgery and Anaesthesia}

The following surgical procedures were performed on the tree shrews from whom data were collected for chapters 5 and 6. Surgical procedures are as outlined in the Methods chapter. Briefly, the animal was anaesthetized using a mixture of Ketamine and Xylazine, a venous catheter was inserted in to the femoral vein and a tracheostomy performed to assist in breathing during the experiment. The animal was administered muscle paralysant (Vecuronium Bromide) intravenously and was anaesthetised using Isoflurane (0.5-1\%) for the duration of the experiment. Hard contact lenses were fitted to the eye to prevent corneal drying. In some tree shrews, additional lenses were used to correct for any refractive errors. A craniotomy and durotomy were performed over the location of V1 (Horsley-Clarke Co-ordinates A2.5 to P2.5). ECG and frontal EEG were monitored during the experiment. At the end of the experiment, the animal was euthanized using an overdose of pentobarbital sodium and perfused using 0.1M Phosphate Buffer (PB) solution followed by 4\% Paraformaldehyde in 0.1M PB. The brain was removed and stored in sucrose (20-25\%) for histology.

	\subsubsection{Electrophysiology}
High impedence, lacquer coated tungsten microelectrodes (FHC Metal Microelectrodes Inc., ME, USA; impedance= 12-18 MΩ) were lowered into the brain at an angle perpendicular to the cortical surface. The signal was amplified and filtered (x 10,000 gain, bandpass filtered between 300-3000 Hz, A-M systems) and fed into an audio speaker as well as an analog to digital converter (Cambridge Electronic Design Limited, Cambridge, UK; digitised at 22.5 kHz). Neurons were recorded from Layers 2/3 and Layer 4. Layer 4 could be identified by a characteristic ‘swish’, first for on stimuli and then for off stimuli, in the tree shrews. Where we no longer heard the swish, we concluded that we exited layer 4 and into layer 5. Neurons in layers 5 and 6 were not recorded from. Lesions (6 μA for 6s) were made at the end of each track. The electrode was withdrawn and lesions were made at regular intervals to trace the path of the electrode through the brain. The data was recorded as a spike trace using the spike 2 software (CED, Cambridge, UK). The spikes were templated and the spike timing exported as a text file. Further analysis was performed using custom MATLAB code (The Mathworks Inc, USA).
\subsubsection{Stimuli}
A hand-held projectoscope was used to mark the receptive field boundaries. Using this, the centre of the monitor was aligned with centre of the receptive field prior to stimulus presentation. Stimuli were presented using a BARCO monitor (Frame Refresh Rate= 80 Hz; Reference Calibrator Plus; Barco Video and Communications, Belgium) and generated using Visage (VSG, Cambridge Research Systems, Cambridge, UK) and custom Stimulus Description Language (SDL) scripts. The monitor had a mean luminance of 32.6 cdm-2. While recording, the monitor was placed at a distance of 114 cm from the eye. For each of the different stimuli described below, ten complete stimulus presentations were completed.
\paragraph{Bar Stimuli}
For each neurons, an initial estimate of optimum orientation was obtained using bars, moving bi-directionally across the screen. The background was a uniform gray screen. Depending on the polarity of the neurons, either a bright bar or a dark bar was used (contrast= 100 \%). The bar was usually 8$^o$o long (ranging between 4 and 8 degrees) and 0.5$^o$ wide (ranging between 0.1 and 1 degree). A total of 18 different orientations were tested and PSTHs (see chapter 2) were made online using the Spike 2 software. The orientation that yielded the highest firing rate was used for further testing.

\paragraph{Grating Stimuli}
For all neurons, once optimum orientation was determined, spatial frequency tuning of the neurons were studied. Drifting sine-wave gratings (TF= 4Hz, Contrast=100\%) of increasing spatial frequencies (between 0 and 2.2 cpd) and in the optimum orientation were presented to neurons. Further, the spatial frequency response of the neuron to gratings tuned to the orientation orthogonal to the optimum orientation were also recorded. The responses were recorded and stored for further analysis.

\subsubsection{Data Analysis}

\paragraph{Orientation Selectivity of bars}

The orientation selectivity of all the cortical neurons we encountered were measured using thin bars. The circular mean and circular variance of this response was calculated using the following formulas to measure the optimum orientation and sharpness of the tuning.

Circular mean=

Circular Variance=

One of the key predictions of our model was that the optimum orientation of the neuronal response does not vary along a penetration perpendicular to the cortical surface. In order to check this, we calculated the absolute difference in preferred orientation between the first neurons we encounter in layer 2/3 in each track and all the neurons that are present in the same track.

While making electrode tracks, due to the angle of the skull and the brain, it is possible that in some of our penetrations, the electrode angle was not always exactly perpendicular to the skull. In order to make sure that any differences we observed were not due to the angle of the track, we also undertook a simulation. We obtained an orientation tuning map in the tree shrews (Bosking et al., 1997) and 
\subsubsection{Histology and Track Reconstruction}


\section{Results}

\subsubsection{Laminar Position of neurons}

The laminar position of all units were determined using track reconstructions   
\section{Discussion}
\section{Conclusion}