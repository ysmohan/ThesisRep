\chapter{Radial bias in the inputs to the cortex}
	\section{Introduction}
	
	\subsubsection{Radial Bias}
	\subsubsection{Sub-cortical biases}
	\subsubsection{Whaat??}
	
	
	\section{Specific methods}
	\subsection{General and surgical procedures}
	
		Five male macaques (Macaca nemestrina; 2-4 years of age) were used in this study, which was
	approved by the University of Melbourne/ Florey Institute of Neuroscience and Mental Health
	Animal Ethics committee and conformed to the guidelines of the National Health and Medical
	Research Council’s Australian Code of Practice for the care and use of animals for scientific
	purposes. After the initial induction of anaesthesia with Ketamine (Ketamil 15mg/Kg i.m.,
	Parnell Laboratories, Australia) and Xylazine (2 mg/kg i.m., Troy Laboratories, Australia),
	cephalic veins of both forelimbs were catheterized and the trachea cannulated for artificial
	ventilation. Anaesthesia was maintained with Isoflurane (0.5-2%) in a mixture of Nitrous oxide
	and oxygen (70:30) throughout the experiment. Skeletal muscle paralysis was induced with a
	bolus of Vercuronium (0.7 mg/kg, Organon Australia Pty Ltd) and maintained with a mixture of
	Vecuronium (0.2 mg/kg/hr) in 4\% glucose and 0.9\% saline through one of the venous catheters.	End tidal carbon-di-oxide was maintained at 3.6-3.8\%. Electrocardiogram (ECG) and
	electroencephalogram (EEG) were recorded to assess the depth of anaesthesia and the dosage of
	Isoflurane was adjusted to maintain an adequate level of anaesthesia. Core body temperature was
	monitored using a sub-scapular thermistor, which also provided feedback to a servo- controlled
	blanket to maintain the body temperature at 36-37 degree Celsius. The eyes were dilated using
	topical administration of 0.1\% Atropine (Sigma Pharmaceuticals Pty Ltd, Australia). Rigid gas
	permeable lenses were inserted to prevent the eyes from drying and appropriate optical lenses 
	and 4mm artificial pupils were used to correct the refractive error and minimize optical
	aberrations. Craniotomy (Horsley-Clarke coordinates: 24-34 mm posterior and 2-10 mm lateral)
	and subsequent durotomy were performed to expose the dorsal aspect of the occipital lobe
	corresponding to the macaque primary visual cortex (V1). A metal chamber (of diameter 10mm)
	was mounted over this opening using dental cement (Dentimex VA, Netherlands) and the
	chamber was filled with silicone oil (Poly methyl siloxane 200, Sigma Australia) and sealed tight
	with a cover glass.
	\subsection{Stimulus}
	Visual stimuli were generated by a Visage stimulus generator (SDL, Cambridge Research
	Systems, UK) at 80 Hz on a BARCO monitor (Reference Calibrator plus; Barco Video and
	Communications, Belgium) at 57 cm from the animal. Stimuli were full-field, high contrast,
	square-wave gratings (1-4 cycles/deg moving at 1-1.5Hz) and presented in 8 different
	orientations drifting in one direction and then the other. Each grating stimulus was presented for
	7.2 s with an interstimulus interval of 10s between gratings when the animal viewed a blank
	screen. Data was collected for 50 complete presentations of each stimulus.
	\subsection{Optical Imaging of intrinsic signals}
	Optical Imaging of intrinsic signals12 was used to obtain stimulus response maps from the dorsal
	surface of the primary visual cortex (V1) using an imaging system (VDAQ Imager 2001, Optical
	Imaging, Rochester, NY). The cortical surface with its vascular details was initially imaged with
	an optical closed circuit camera (Teli CS8310B, with tandem optics: 2 x Pentax lenses, f=50mm)
	under green illumination (545 nm). This high contrast image (green image) with the vascular
	landmarks (Fig 1A) was used as a reference image to align the response maps of the cortex.
	Intrinsic signals were acquired using a 630 nm red light focussed at 550-700 μm below the
	cortical surface. The cortical region for imaging was selected to include a fairly flat surface, not
	only to enable good OI conditions, but also to precisely overlay the topographical map of the
	visual field on the cortical surface from the microelectrode recordings. For each stimulus
	condition, 18 data frames, each 400 ms long, was generated. The signal-to-noise ratio was further
	enhanced by averaging 50 trials for each stimulus.
	\subsection{Topographical recordings}
	Following imaging, electrophysiological recordings using lacquer coated tungsten electrodes (6 -
	12 MOhms, FHC Inc, ME,) were performed at different topographical locations across the
	imaged area. The signals were amplified (10000x) and filtered (300-3000 Hz) using an amplifier
	(AM Systems model 1800, Washington). The minimum discharge fields/ receptive fields of
	encountered neurons were carefully hand-plotted on a wall chart. A back projecting fundus
	camera was used at regular intervals to plot reference landmarks of the eye (optic disc
	parameters, vasculature and the foveal centre) to account for any ocular drifts over the course of
	the experiment and obtain the precise radial angle of the recorded neurons.
	\subsection{Data Analysis}
	Data analysis was performed using custom scripts generated in MATLAB. Stimulus response
	maps were obtained as an average of 14 data frames (frames 3-16) followed by first-frame
	subtraction to remove illumination artefacts across 50 trials for each stimulus condition. The
	optimum orientation of individual pixels was calculated by vector-averaging the pixel values
	from the stimulus response maps29. We refer to the computed map without any image
	manipulation as the “veridical” map. The conventional orientation map was also generated by
	methods12,29. Briefly, the stimulus response maps were filtered using a difference of Gaussian
	method to isolate features between sigma= 312.5 um and sigma= 100 um. The pixels from the
	resultant single condition maps were then vector-averaged to calculate their optimum
	orientations and provide the “filtered” maps (See Supplementary Figure 1).
	We used the receptive field locations in reference penetrations, projections of the foveal centre
	using the fundus camera and cortical magnification factor values across the retina25
	(Supplementary Figure 3) to determine the radial angle of points on the cortical surface that
	were uniformly spaced (375 μm apart). We then calculated the average orientation response of
	pixels in a 750 μm square (optimum orientation) around each of the points for both the filtered
	and veridical conditions. We compared the computed radial angle and the optimum orientation to
	determine what extent of the optical imaging response preferred stimuli close to the radial
	orientation. We also vector-averaged the response of each pixel and collected the data from all 5
	animals in one histogram (Figure 2b).

	\section{Results}
	
	We imaged an exposed area of the primary visual cortex in five anaesthetised macaques (in
	three, left V1 and in two the right). When no band-pass spatial filter was applied to the
	haemodynamic signal and we created a veridical (unfiltered) orientation map (Supplementary
	Figure 1), we found that large areas of the cortex were tuned to a narrow range of orientations
	(see Figure 1 for one animal and Supplementary Figure 2 for maps from the other four
	animals). In order to accurately quantify whether a preference for the radial orientation is present
	in the cortical OI maps, we made multi-unit recordings from microelectrode penetrations made
	perpendicular to the cortical surface (Figure 1 and Supplementary Figure 2) and obtained a
	map of receptive field locations within the imaged area. For 750 x 750 μm areas (30x30 pixels)
	spaced 375 μm apart (Regions of Interest or ROIs; see Supplementary Figure 3), the radial
	orientation was calculated using receptive field locations of at least 4 reference penetrations for
	each animal and published values of cortical magnification factor25. Preferred orientation of the
	signal was calculated from the OI data by taking the mean of the individual pixels within an ROI.
	This was done separately for veridical and filtered orientation domain maps. The absolute
	difference between the radial orientation of the visual field locus of the ROI and optimum
	orientation of the intrinsic signal of the ROI was calculated for the filtered and unfiltered
	conditions for all 5 animals (n=495) and these are shown as a histogram in Figure 2a. The
	distribution for the veridical and filtered maps were significantly different (χ2=513.5; df=3;
	p<0.0001). For the veridical maps, majority of ROIs were tuned to the radial orientation and the
	distribution was significantly different from a uniform distribution (χ2=718.1; df=3; p<0.0001).
	For the filtered maps, although the distribution of absolute differences was significantly different
	from a uniform distribution (χ2=71.2; df=3; p<0.0001), no preference for radial orientation was
	evident. The analysis was repeated for individual pixels (n= 135,122) instead of for the (30x30
	pixel) ROIs. The results (Figure 2b) were similar. The distributions of the veridical and filtered
	maps were significantly different (χ2=28749; p<0.0001). For the unfiltered maps, majority of
	pixels were tuned to the radial orientation and the distribution was significantly different from a
	uniform distribution (Figure 2b, χ2=13590.2; p<0.0001). For the filtered maps, the distribution
	of absolute differences was significantly different from a uniform distribution (χ2=4004;
	p<0.0001), but there is a hint of preference for radial orientation and its orthogonal.
	[If accepted, we intend to shift this paragraph to Supplementary Information]. With the above
	analysis, due to the large sample sizes, both the veridical and the filtered distributions were seen
	to be significantly different from a uniform distribution, whereas as per our hypothesis, the effect
	should be much stronger for the veridical samples. This difference can be observed in the
	histograms shown in Figure 2. However, in order to control for the effect of sample size, we
	randomly sampled 1000 times (1000 trials) with replacement from the original data sets of the
	veridical and filtered maps. We set two sample sizes - 40 (see Supplementary Figure 4) and
	1000 (see Supplementary Figure 5). Individual lines in a and b in these figures show the
	distribution of the data points with respect to the radial orientation (set as 0 degrees) for each
	trial in the filtered and the veridical conditions respectively. The distribution of the chi-square
	test statistic obtained for the 1000 trials is shown in panels c and d for filtered and veridical
	conditions respectively. When the sample size was set to 40, the individual pixels in the filtered
	condition were uniformly distributed (Mean χ2 over 1000 trials =8.27, 95% confidence interval =
	[7.8, 8.36 ]) but the individual pixels in the veridical condition were significantly different from a
	uniform distribution (Mean χ2 over 1000 trials =46.11, 95% confidence interval = [45.83, 46.4]).
	When sample size was set to 1000, the same pattern was observed for the pixels sampled from
	the veridical maps as has been reported above (Mean χ2 over 1000 trials =1012.84, 95%
	confidence interval = [1012.1, 1013.59]). However, for the filtered maps, with the distribution
	significantly different from a uniform one (Mean χ2 over 1000 trials =36.73, 95% confidence
	interval = [35.98, 37.48]), smaller peaks at the radial and the orientation orthogonal to the radial
	were observed, This is consistent with the radial biases observed in cortical recordings and in
	human psychophysics21-23.
	
	\section{Discussion}
	We believe that the orientation selectivity and the radial bias observed in the veridical maps
	reflect the biases in the thalamic inputs to the cortex and the subthreshold activity within the
	cortex. If most thalamic inputs to cortex are indeed tuned to one of the cardinal orientations, it
	has important implications for the development of cortical architecture. It is very likely that the
	signals conveying information on stimulus orientation arrives in the cortex in a small number of
	broadly tuned channels and that the cortex develops the whole range of orientation selectivities
	from these inputs, similar to the well-known instance of primate colour vision3. Such a
	mechanism giving rise to a gradual change in orientation preference in the form of the classical
	orientation domain architecture3 is comparable to the organisation demonstrated for other
	cortical features, especially for ocular dominance1 and ON/OFF domains26. Though there is only
	a mild tendency for a second cardinal orientation in our data, this may not be necessary, since we
	know that in colour vision, the ratio of L and M cones vary hugely between individuals with
	normal trichromatic vision, who nevertheless have very similar colour vision27. Our study also
	establishes the value of observing the OI signal at different spatial scales, as recently
	demonstrated for macaque area V428.