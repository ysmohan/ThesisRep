\chapter{extras}


Our results are consistent with those published by Van Hooser et al., 2013.



Hubel and Wiesel showed that t
Read Daugman.

Eccentricity of the recording sites: we have a range= $<$10$^o$ azimuth but this may not be accurate enough? Perhaps look at just one track? Show results in results section? 
Get average bandwidth in octaves to be able to compare to other species.
Narrow tuning widths?
Patch by patch analyser?
Average properties of simple cells resemble edge or line detectors- but if this is the case they can't actually demonstrate linear summation over their receptive fields.

Psychophysical papers?


Plan:
1) Summary of results
Differences in modulation index and segregation index. What this means? Linearity of neurons?
Segregation index: no neurons that had completely overlapped sub-regions i.e. si<0.2.
with the rest of the SI, evenly distributed across the layers.
There was no significant differences between layers.

Modulation index: Although not statistically significant, modulation index<1 for most layer 2/3 and layer 3c. modulation index>1. Most layer 4 neurons, have a modulation index between 1 and 1.2. This is the equivalent of between 1 and 1.57 using the standard modulation ratio calculated (F1/F0). These neurons still have a higher modulation index but not high enough. Could be potential B cells described by Henry et al or the non-linear simple cells described by other people. 

Simple cells are found in input layers while complex cells are found in supragranular layers. True if we look at the modulation index but not when looking at the segregation index. Provides support against the heirarchical model of visual processing where simple cells project to complex cells. Also has been shown in other species- complex cells are found in layer 4 and simple cells in supragranular layers. We see the same trend here.

How do our results of segregation index and modulation index compare with the previously published results for segregation and modulation ratios?  Our results are similar to previously published results by Van Hooser et al., 2013. Both results seem to show a unimodal distribution with a range of linearities in the receptive fields when compared to a simple/complex bimodal distribution. Is this because of the measure used for modulation index? Checked with regular modulation index (F1/F0) This measure also did not yield a bimodal distributions.

What does the relative bandwdth and spatial frequency relationship mean?

We found that in most cases, there was a negative relationship between the pk spatial frequency and the relative bandwidth of the neurons, especially when the linear component of all the neurons were used to run the analysis. This means that most neurons in the shrew V1 actually do act as linear filters in optimum range of the neurons (See Fig.\ref{fig:fig1}).
What exactly does this mean? The cortex could be completely throwing out this information when non-linear?

What is linearity even useful for?

ARe there simple and complex cells in the shrew cortex? Does this mean anything?


Further, it has also been suggested that linearity is not a requisite feature of simple cells. Neurons in the LGN maybe classified as X, Y and W cells. While X cells show linear sustained responses, Y cells exhibit transient, non-linear responses. While originally thought that X and Y cells projected to simple and complex cells respectively, this connection has since been disproved. As a result, significant non-linearities may be introduced in simple cells depending on the type of input that they receive. Further, if simple cells do function as edge detectors rather than linear filters, they are unlikely to be linear neurons (DeValois and Webster, 1978). Hence, alternate methods of classifying simple cells are also examined below. 
Whether there are cortical simple and complex cells have also been debated. Depending on stimulus parameters, there seems to be a continuum of neurons rather than a bimodal distribution of neurons in the primary visual cortex. So the linear component of all neurons have also been subjected to the same analysis.