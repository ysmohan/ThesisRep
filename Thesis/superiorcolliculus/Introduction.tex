\documentclass [12pt]{report}

\usepackage{graphicx}
\usepackage{setspace}
\doublespacing

\begin{document}
\chapter{Orientation tuning in the Tree Shrew superior colliculus}
\pagebreak
\section{Abstract}

Though theories of orientation selectivity suggest that orientation biases observed in V1 inputs are the result of excitatory convergence, studies have shown that bias in the inputs may be inherited from neurons in sub-cortical structures, especially the retina and the lateral geniculate nucleus (LGN). Congruent with this theory, retinal and LGN neurons have been shown to be tuned to orientation at higher spatial frequencies. If orientation selectivity arises from the retina, it should be evident in other targets of retinal projections. The superior colliculus (SC) is one such area. Here, I examined the orientation selectivity of SC neurons in tree shrews using thin bars and gratings of various spatial frequencies. I found that SC neurons show orientation tuning comparable to that observed in layer 4 of V1 in the tree shrews and orientation biases reported in the retina and the LGN of cats and macaques. This orientation selectivity was more evident at higher spatial frequencies. These results indicate that orientation tuning observed in the inputs to the cortex maybe generated from the orientation biases present in earlier visual areas.
\pagebreak
\section{Introduction}

The tree shrew superior colliculus is a large well laminated area of the brain. Tree shrew studies played an important role in the discovery of the functional subdivisions of the superior colliculus, one of which receives visual inputs and projects to extra-striate visual cortices. This pathway to the visual cortex via the superior colliculus forms an alternate pathway to the geniculo-cortical pathway. Considering the important role superior colliculus plays in vision, surprisingly little is known about the receptive field properties of the tree shrew superior colliculus. Unlike the geniculo-striate system, the superior colliculus studies are also fraught with interesting species differences. In this study, we examine the receptive field properties of the visual neurons in the tree shrew superior colliculus. In particular, we examined the orientation and spatial frequency tuning of the tree shrew SC neurons to elucidate the source of orientation tuning.

The superior colliculus is essential for form discrimination. For a long time, the superior colliculus was only implicated in oculomotor behaviour (Sherrington, 1947). This is due to its reciprocal projections with multiple sensory as well as the motor areas of the brain (References). Lesion studies conducted in the superior colliculus  showed that the superior colliculus consisted two separate systems --- the superficial layers involved in visual form discrimination and the inferior layers implicated in oculomotor function and orienting behaviour (Casagrande et al., 1972). The superficial layers of the SC received visual inputs from the retina and also formed reciprocal connections with other thalamic visual nuclei, namely the dorsal lateral geniculate nucleus and the pulvinar.

The superficial layers of the superior colliculus then form a separate visual pathway to the visual cortex via the pulvinar. Histological studies have shown that an anatomical pathway from the retina to the superior colliculus to pulvinar to extrastriate cortical areas exist (Harting et al., 1973). Lesion studies show that form perception is preserved when striate cortex is lesioned. When the temporal cortex is lesioned, the tree shrews were incapable of adapting to changing cues. The functional deficits demonstrated by lesioning the temporal cortical areas are also seen when the visual areas of the superior colliculus are lesioned. This preservation of form discrimination in the shrews following lesions to the visual cortex is presumably due to the alternate pathway to the extrastriate regions via the superior colliculus. This pathway has also been implicated in "blindsight" where humans demonstrate visual functionality following V1 lesions, eventhough they have no visual percept (see Payne et al., 1996 for review).

Although the tree shrew superior colliculus is functionally well characterised, there is little known about the receptive fields of the superior colliculus. Only one study characterised the receptive fields of the tree shrew superior colliclus. Albano et al (1978) examined the receptive fields of the tree shrew superior colliculus. Unlike receptive fields previously reported in the visual regions of the superior colliculus in cats and macaques, Albano et al reported that the tree shrew superior colliculus showed poor direction selectivity and a small proportion of neurons (~20\%) were tuned to orientation (aspect ratio $>$ 3). These properties have not been further studied in the tree shrew superior colliculus.

More than 50 years after its first report, the mechanism underlying orientation selectivity is still debated. The theory of excitatory convergence of orientation selectivity suggests that orientation selectivity first originates in the primary visual cortex. In line with this theory, the orientation biases that have been demonstrated in sub-cortical areas have been largely ignored. Therefore, sub-cortical receptive fields in the tree shrew are still thought of as unoriented (Scholl et al, 2014). A recent study in the tree shrew geniculostriate system showed that LGN neurons of the tree shrew showed orientation biases (Van Hooser et al., 2014). While there are not many reports of orientation selectivity in the SC of cats and macaques, studies in the mouse superior colliculus indicate that neurons located here are biased for orientation (eg: Ahmadlou et al., 2015). In this chapter, orientation biases of the superior colliculus neurons in the tree shrew further explored.

The superficial layers of the tree shrew superior colliculus receive inputs from the retina as well as the primary visual cortex. As mentioned earlier, the superficial layers of the tree shrew play an important role in form discrimination. These layers are stratum zonale(SZ), stratum griseum superficiale (SGS) and stratum opticum. The SZ predominantly consists of fibres. The SGS is further subdivided into upper and lower SGS. The upper and lower SGS of tree shrews are morphologically different. The uSGS of tree shrews have larger neurons and predominantly receive inputs from the retina. The lSGS contains smaller neurons and receive inputs from both the retina and V1. The SO contains a combination of neurons and fibres and receives inputs from the retina, V1 and the SGS. The response properties of the SGS and SO will be examined in this chapter.

Since SC neurons receive inputs from both retina and the primary visual cortex, an additional aim of the study was to elucidate the source of orientation selectivity of superficial SC neurons. Neurons of the primary visual cortex show sharp orientation selectivity and a bandpass spatial frequency tuning. Retinal neurons are broadly tuned to orientation and have a low pass spatial frequency tuning (Levick \& Thibos, 1982). Further, retinal neurons show sharper orientation tuning at higher spatial frequencies; that is at higher spatial frequencies, the high spatial frequency cut-off at the optimum orientation is higher than the high spatial frequency cut-off at the orientation orthogonal to the optimum orientation (orthogonal orientation). The LGN neurons of cats then reflect a similar pattern of orientation and spatial frequency tuning observed in the retina (Vidyasagar \& Heide, 1985). Here, we examined the orientation and spatial frequency responses of the neurons. We hypothesise that:

a) Superficial SC neurons will show oriented responses when shown thin bars.

b) Superficial SC neurons will have low pass spatial frequency tuning.

c) At higher spatial frequencies, the superficial SC neurons will show sharper orientation tuning.

\section{Methods}

\subsection{Surgery and anaesthesia}

Surgical procedures have been outlined in chapter 5. Briefly, the animal was anaesthetised, a venous catheter was inserted in to the femoral vein and a tracheostomy performed to assist in the breathing. The animal was administered muscle paralysant through the catheter (norcuron) and was anaesthetised using Isoflurane (0.5-1\%). A craniotomy over the location of the primary visual cortex was made and a durotomy was performed. EEG and ECG were monitored during the experiment.
\subsection{Electrophysiology}
 High impedence, lacquer coated tungsten microelectrodes (FHC Metal Microelectrodes Inc., Bowdoinham, ME, USA; impedance= 12-18 M$\Omega$) were lowered into the brain and the signal was amplified and filtered (x 10,000 gain, bandpass filtered between 300-3000 Hz, AM systems amplifier) and fed into an audio speaker as well as an analog to digital converter (CED, Cambridge Systems, digitised at 22.5 kHz). The SC was identified by listening to the neuronal activity in the speaker. The data was recorded as a spike trace using the spike 2 software. The spikes were templated and the spike timing exported as a text file. Further analysis was performed using custom MATLAB code.

\subsection{Stimuli}

A hand held projectoscope was initially used to demarcate the receptive field boundaries. Using this, the centre of the monitor was aligned with centre of the receptive field prior to stimulus presentation. Stimuli was presented using a Barco Reference Calibrator Plus monitor (Barco monitor; Barco Industries, Belgium, Frame Refresh Rate= 100 Hz) and the stimuli were generated using Visage (VSG, Cambridge Research Systems, Cambridge, UK) and custom Stimulus Description Language (SDL) scripts. The monitor had a mean luminance of 32.6 cdm$^{-2}$. In some experiments, an antiglare, anti static screen was used. The luminnance when this screen was used was 17.4 cdm$^{-2}$. The monitor calibration was regularly checked using the PR-650 spectrophotometer (Photo Research, Palo Alto, CA, USA). While recording, the monitor was placed at a distance of 114 cm from the eye.

For each SC neuron, the preferred stimulus orientation was initially measured using a thin moving bar. The bar was presented in 9 different orientations sweeping bi-directionally (a total of 18 orientations.). The background was a uniform gray screen. Depending on the polarity of the neurons, either a bright bar or a dark bar was used (contrast= 100 \%). The bar was usually 8 $^{o}$ long (ranging between 4 and 8 degrees)and 0.5 $^{o}$ wide (ranging between 0.1 and 1 degree). The velocity of the bar was between 5 and 20 $^{o}/$second.

Peri-stimulus-time-histograms (PSTHs) were generated online using the spike 2 () software. Based on the PSTHs generated following the presentation of the bar, the optimum orientation of the bar was determined as the orientation that gave the maximum response. This orientation was used for further testing.

The spatial frequency response to gratings were then measured. The animals were presented with drifting sine-wave gratinges of varying spatial frequencies (TF= 4Hz, SF= 0 cpd to 2 cpd) at 4 different orientations (optimum, optimum + 90$^{o}$, optimum+45$^{o}$, optimum-45$^{o}$). In some cases, responses to a complete orientation tuning stimulus (16 directions/ 8 orientations) were recorded in order to further quantify the orientation response at a certain spatial frequency.

\subsection{Data Analysis}

Regardless of the stimulus presented, the following analysis was performed on the extracellular trace before any specific analysis. Spikes were templated based on their polarity, size and timing. The spike time and stimulus marker exported into text files. Using custom scripts in MATLAB (see Appendix), peri-stimulus-time-histograms (PSTHs) were constructed for each of the stimulus conditions. Spike density functions were created using a 3 bin moving average function. This SDF was used for further analysis.

For orientation tuning recorded using a bar, the peak response in the SDF for each direction was plotted on a polar diagram. The circular mean of this maximum response and the corresponding direction was calculated using the following formula:


The circular variance (CV) and the orientation selectivity index(OSI) were also calculated as follows:

CV=

OSI=

For the gratings, the Discrete Fourier Transform (DFT) of the spike density function was calculated using the MATLAB fast fourier transform algorithm. The F1 and the F0 component were calculated as mentioned in the general methods. The F0:F1 ratio was calculated at the peak spatial frequency. The peak spatial frequency is the maximum spatial frequency after which both the F0 and the F1 decrease. If the F0 response was smaller than the F1 response (ie. the ratio was less than 1), the cell was deemed to be X- like and the magnitude of the first harmonic component of the repsonse was used for further analysis. If the ratio was greater than 1, the cell was considered non-linear and the F0 component was used.

The spatial frequency tuning at the optimum and orthogonal orientations were calculated by linearly interpolating between the data points. The bandwidth during which the superior colliculus neurons responded for the optimum orientation but not for the orthogonal orientation was calculated. In order to do this,  a minimum response was first defined as the response rate at the spatial frequency where the response between the optimum and orthogonal orientations were no longer significantly different. The spatial frequency where the response rate for the optimum and orthogonal orientations first reach the minimum response was termed the optimum SF cutoff and orthogonal SF cutoff. The difference between SF cutoff for the optimum and orthogonal spatial frequencies were calculated.

	\section{Results}
	\subsubsection{Anatomical location of units}
	
	A total of 22 units (5 tracks in  3 Tree Shrews) were recorded from. The laminar position of all the units were determined by reconstructing the electrode tracks using electrolytic lesions. The photomicrograph from one of the Nissl stained sections in one of the tree shrews is presented in figure 6.1a. In this section, lesions made in 2 separate tracks are visible (red arrow points to one of them). The different layers of the tree shrew SC are marked. The superficial layers are further distinguished. Electrode reconstruction was completed in all animals and the laminar position of each of the neurons is shown in Figure 6.1b. All the neurons we recorded from were located in the superficial layers with the majority being in the Stratum Griseum Superficiale (SGS) where the majority of the retinal inputs terminate.
	
	\begin{figure}
		
		\includegraphics[width=\linewidth]{anatpos.jpg}
		\caption{Histology. a) A section of tree shrew superior colliculus showing electrolytic lesions. Red arrow
			points to an electrolytic lesion. Scale bar (yellow vertical line) denotes 1000 μm. b) A summary of laminar
			position of recorded units in the superior colliculus. Abbreviations: uSGS- upper Stratum Griseum Superficiale;
			lSGS- lower Stratum Griseum Superficiale; SO- Stratum Opticum.}
		\label{fig:fig1}
	\end{figure}
	
	
	
	\subsubsection{Orientation Selectivity}
	
	The response of a representative neuron to moving bars of different orientations and the corresponding orientation tuning curves are presented in figure showed in figure 6.2. The response was the average of 10 trials and the error bars are $\pm$ sem. The CV of this neurons was 0.82. The median CV of all the neurons in our sample was 0.82 with a range of [0.29, 0.94]. Any neuron with CV greater than 0.9 was considered not selective to orientation. Two neurons had a CV greater than 0.9 and were excluded from further analysis. The orientation tuning curves of the most selective, least selective neuron with Cv less than 0.9 and the least selective neuron in the entire sample are presented in figure 6.3. The histogram of all the circular variances are presented in figure 6.4.
	
	\begin{figure}
		
		\includegraphics[width=\linewidth]{SCOriResp.jpg}
		\caption{Orientation response of an example cell. a) The polar plot of the orientation responses of a neuron in the tree shrew superior colliculus. Each spoke represents an orientation presented. The circular variance of this neuron is 0.82. This was also our median circular variance. b) The spike density functions for different orientations for the neuron whose polar plot is shown in a. The orientation and direction of movement of the bar is shown above the trace.}
		\label{fig:fig2}
	\end{figure}
	
	
	\begin{figure}
		\includegraphics[width=\linewidth]{rangeoritun.jpg}
		\caption{Polar plot showing
			the orientation tuning of the bar. Error bars denote
			Standard error. Orientation tuning curves of the
			sharpest (a) and the least tuned (b) neurons
			included in our analysis. (c) was the least tuned
			neuron in our sample}
		\label{fig:fig3}			
	\end{figure}
	
	
	\begin{figure}
		\includegraphics[width=\linewidth]{cvlampos.jpg}
		\caption{Circular variances: (a) This figure shows the distribution of circular
			variances of all neurons. Most of the tuned neurons have a CV between 0.7 and 0.9. The apparent second peak is discussed further in the discussion (b) This figure shows the laminar position of the individual neurons plotted against the circular variances. Apart from the three neurons in the upper SGS that are sharply tuned to orientation, there doesn't seem to be any differences in the orientation selectivity between the upper and lower SGS. There was an inadequate sample from the SO for comparison.}
		\label{fig:fig4}			
	\end{figure}
	
	An additional measure of orientation selectivity, the Orientation selectivity index (OSI) was also calculated. This was done predominantly to enable comparisons to previous studies in both tree shrews as well as other species.As has been previously shown, the orientation bias is a reciprocal measure of orientation selectivity compared to the circular variance. Lower values of bias indicate that the neurons are more broadly tuned and once again, there seem to be two distinct groups of neurons, ones that have a fairly low orientation bias and a small group that has sharper orientation tuning.
	
	\subsubsection{Spatial Frequency Tuning}
	
	A summary of the results of spatial frequency tuning we obtained from 16 neurons is presented in figure 1.5a. The median peak spatial frequency was 0.2 cpd (range=0.6 cpd) and the median half width at half height was 0.35 cpd (range= 0.65 cpd). Although most neurons reached their peak firing rate quickly, they tended to fire over a range spatial frequencies as indicated by the slower rise of the high spatial frequency cut off curve. Another important point to note, is that nearly 80\% of the superior colliculus neurons were low-pass tuned to orientation (12/16, significantly different to chance, p= 0.028).
		
		\begin{figure}
			\includegraphics[width=\linewidth]{cumsum_sf_SC_LGN.jpg}
			\caption{Cumulative sum of the low-cutoff, optimum and high cut off spatial frequencies in the tree shrew Superior Colliculus (left) and the Lateral Geniculate Nucleus (right; LGN). The LGN results were published in the paper by Van Hooser et al., 2014 and the right side panel is from figure 7b, plotted on the same scale as the SC data in the left hand side panel.}
			\label{fig8:fig8}
		\end{figure}
	In order to enable a direct comparison between the superior colliculus and the lateral geniculate nucleus, data from the LGN (from Van Hooser et al., 2014, Figure 7b, 30 neurons) is plotted next to the superior colliculus data. LGN cells tended to have a higher peak spatial frequencies and lower high frequency cut-offs. However, a similar proportion of neurons are bandpass tuned when compared to the superior colliculus.
	
	When the spatial frequency tuning response of the neuron at different orientations was observed, 13 of 16 neurons were orientation tuned at higher spatial frequencies. The spatial frequency response of an example neuron at the optimum and the orthogonal orientations is presented in figure 6.5a. The response is the F0 component of the FFT. The gray shaded area represents the spatial frequnecies where the neuron still responds to the optimum orientation but no longer responds to the orthogonal orientation (ie. the neuron is orientation tuned).  The upper limit of the gray shaded area (the dotted line to the right) is the cut off spatial frequency at the optimum orientation. The sf corresponding to the lower limit of the shaded gray area is the cut off spatial frquency. The difference in response between the optimum and non-optimum orientation cut off frequencies was calculated. These results for the group are presented in figure 6.6 a. On average, the response to the orthogonal orientation reached the minimum 0.5 cpd before the response to the optimum orientation; with the 95 percent CI= [0.4, 0.6].
	
	The OSI at each of the spatial frequencies for the example neuron is plotted in figure 6.5 b and the group results are presented in figure 6.6 b.
	The neuron exhibited the highest bias close to the cut off frequency at the orthogonal orientation.

	\begin{figure}
		\includegraphics[width=\linewidth]{SCOptOrth.jpg}
		\caption{Example SF tuning curves for optimal and orthogonal orientations. The cut-off frequency at the
			optimal orientation is the SF at which the response at optimal orientation is no longer significantly different from the response at orthogonal
			orientation. The response at the cut-off frequency for optimum orientation is called the minimum response. For the orthogonal orientation, the
			cut-off frequency was the SF at which minimum response was first reached.}
		\label{fig:fig5}			
	\end{figure}
	
	\begin{figure}
		\includegraphics[width=\linewidth]{SCSFTuning.jpg}
		\caption{ The difference between the cut-off frequencies for the optimum
			and orthogonal orientations for 16 units is shown in Figure 3b.}
		\label{fig:fig6}			
	\end{figure}
	
	\begin{figure}
		\includegraphics[width=\linewidth]{SCvgeniculostriate.jpg}
		\caption{ Comparison of the Superior Colliculus vs neurons in the geniculostriate system (data collected by Van Hooser et al., 2013)}
		\label{fig:fig7}			
	\end{figure}
	
	\section{Discussion}
	The results of this study demonstrate that neurons in the superior layers of the superior colliculus are tuned to orientation at higher spatial frequencies. This finding in combination with other reports of orientation biases in sub-cortical areas renders one of the key assumption of the excitatory convergence model— that subcortical neurons have circular, unoriented receptive fields which then requires the arrangement of their receptive fields in a row to give rise to orientation tuning— incorrect. Not only do tuned cortical inputs then pave the way for intracortical inhibition to sharpen orientation selectivity, they also abet the development of cortical architecture.
	
	\subsection{Anatomical Relevance}
	
	The histology confirmed that all the units that were recorded from the superficial layers of the superior colliculus. While the superior colliculus receives information from all the sensory modalities, the superficial layers receive direct input from the retina and feedback projections from the primary visual cortex. They also project to extrastriate visual areas. Lesion studies have shown that when the shrew SC is lesioned, form perception is affected. In Studies where the primary visual cortex of the tree shrew was ablated while keeping the SC and extra-striate visual areas intact showed that tree shrews could still consciously perceive form information further implicating the superficial layers of the shrew SC in playing an important role in perception. Given its position in this alternate visual pathway and its role in form perception, it is surprising that orientation tuning has not been reported in the Superior Colliculus. Where it has been reported, like in the case of the tree shrews, a very small proportion of neurons have said to be tuned to orientation. These neurons have also been reported in the superficial areas of the superior colliculus. 
	
	\subsection{Comparison with previous tree shrew studies}
	
	In their earlier paper, Albano et al., 1978 suggested that less that 10\% of the neurons had elongated receptive fields. However, in our study, 90\% of our neurons were orientation selective. It is important to make a distinction in these two results. While they may sound like it, these results are not entirely contradictory. In their study, Albano et al tested the elongation of the receptive fields. That is, using the neuronal responses, they plotted the receptive field boundaries of neurons and concluded that any neuron that had an aspect ration of 3:1 had elongated receptive field. In this study on the other hand, we used the response of the neurons to bars and gratings of different orientations. Studies have shown that only a slight receptive field elongation is required for a neuron to give orientation specific response. Albano et al may have simply not detected smaller effects which have been reported in the retina and LGN due to their conservative criterion for classifying a neuron as orientation selective.
	
	Another reason Albano et al., 1978 may not have detected the extent of orientation tuning in the shrew SC could be the stimulus used. As mentioned earlier, bars and gratings were used in this study. Albano et al also used these stimuli however, only one paper was published (1974) in the cat retina indicating that orientation tuning was detected at higher spatial frequencies (Hammond, 1974). However, in the eighties, a lot of papers were published revealing the spatial frequency dependence of orientation tuning. The lack of this knowledge may also be one of the reasons why the orientation selectivity in the superior colliculus was missed.
	
	Van Hooser et al., 2013 published a comprehensive set of data on the transformation of the receptive fields from the lateral geniculate nucleus to the layer 4 (input layer) to layer 2/3 of the tree shrew visual system. The orientation tuning of the superior colliculus neurons are plotted in relation to the geniculate, layer 4 and layer 2/3 neurons in the tree shrew in figure 6.7. This comparison indicates that the orientation biases observed in the superior colliculus are similar to those observed in the LGN of the tree shrew, with approximately 85\% of the SC neurons having similar orientation tuning to the LGN neurons in the Van Hooser study. There is a tendency in our data for around 15\% of neurons to have sharper orientation tuning than those exbhibited in the LGN, closer to those seen in the cortex. While the neurons in the upper and lower SGS receive predominantly retinal inputs, there are also neurons which receive feedback projections from the primary visual cortex. These neurons could be one of the few neurons that receive cortical feedback. This can also be seen in figure 6.4 where the distribution of circular variance seems to be in two different groups. However, the sample size in this study is too small to comment on this segregation.
	
	\subsection{Comparison with previous superior colliculus studies}
	
	The superior colliculus being a large, well laminated organ in most species was intensely studied for a while. The studies conducted in the superficial layers of the cat and macaque superior colliculus showed that the superior colliculus neurons were direcion selective whereas no orientation selectivity was observed. In this sense, the SC was previously compared to the LGN — both sub-cortical areas receiving unoriented input and relaying unoriented inputs to different pathways. However, the realisation that superior colliculus neurons may not be tuned to orientation at higher spatial frequencies seems to have not occured in people who have investigated it. Recent rodent studies have shown that the rodent superior colliculus shows sharp orientation selectivity. Previous studies have demnstrated that the tree shrew superior colliculus is similar to the macaque visual system and the SC makes similar connections to extrastriate cortical areas in the macaque and the shrews. So it is possible that orientation biases are present in these animals as well and will be revealed when tested with higher spatial frequency stimuli.
	
	\subsection {Comparison with the geniculostriate system of cats and macaques}
	One of the prominent paper published investigating the spatial frequency dependence of orientation tuning in the retinal ganglion cells of cats was Levick and Thibos (1982). They characterised the way orientation tuning varied with spatial frequency. In the following paragraph, I will evaluate our results in the context of the responses of retinal ganglion cells.
	
	One of the two key findings of Levick and Thibos was that RGCs were tuned to orientation at higher spatial frequencies. They also found that in some cases, at lower spatial frequencies, the neuron responded better at the orthogonal orientation compared to optimal orientation. They also reported that the degree of orientation selectivity (reported as orientation bias) was the maximum close to the threshold. In the tree shrew SC, all these findings hold true. A close examination of Fig: 6.5 shows that orientation tuning is observed at higher spatial frequencies. Figure 6.5 b also shows that the orientation bias was the maximum close to the threshold. Figure 6.6b also demonstrates this. Figure 6.5 is also only one example of a case where the neuron was biased for the orthogonal orientation at lower spatial frequencies. Neuron being oriented to different orientations at lower spatial frequencies was also a common finding in the superior colliculus. But the optimum orientation of the neuron as measured using bars was the orientation for which the SF cut-off was the highest in all cases. These properties have also been more universally demonstrated in the retina of macaques and also the LGN of cats and macaques, further indicating that the orientation biases have a common, retinal ancestry.
	
	\subsection{Conclusion}
	
	In this chapter, I set out to examine if one of the key assumptions of the excitatory convergence model— that subcortical neurons had circular receptive fields — was indeed true. Previous studies in the retina and the lateral geniculate nuclei of cats and macaques have shown that subcortical neurons were tuned to orientation at higher spatial frequencies. I hypothesised that the tree shrew SC neurons would also be tuned to orientation at higher spatial frequencies. When examined with thin bars and gratings of increasing spatial frequencies, SC neurons were indeed tuned to orientation. These orientation tuned inputs may then be sharpened by intracortical inhibition to generate the sharp orientation selectivity we see in the primary visual cortex. Inputs tuned broadly to a small number of orientations could also give rise to the organisation of cortical columns. Finally, establishing orientation biases in the retina also reduces the functional redundancy of establishing orientation tuning in the different parallel pathways.
	
	

\end{document}