\documentclass [12] {report}

\usepackage{graphicx}
\usepackage{setspace}
\usepackage{here}
\usepackage[a4paper, total={6in, 8in}]{geometry}


\doublespacing

\title{Thesis Summary}

\begin{document}

	
	\tableofcontents
	\chapter{Thesis Summary}
	\section{Problem and Aims}
    Neurons in the primary visual cortex respond selectively to stimulus features. For example, the orientation and the polarity of stimuli, their spatial and temporal frequencies, the eye to which the stimulus is presented (ocular dominance) and the colour of the stimulus. Such features are also spatially grouped together in the primary visual cortex. For example, neurons tuned to the same orientation are grouped together in orientation columns in most species. Neurons that prefer contrast decreases and increases are also organised in columns in the primary visual cortex. While most models of feature selectivity aim to explain either the mechanism underlying feature selectivity or their organisation in V1, few endeavour to explain both. One recent paper suggests a simple solution; that the roots of feature selectivities are established sub-cortically, namely in the retina, i.e., in the retina, broad selectivity for features is established. This feature selectivity is then elaborated by downstream structures.
    
    It is widely accepted that colour selectivity in the visual cortex originates from a limited number of broadband receptors in the retina (the long, medium and the short wavelength cones). However, in V1, neurons demonstrate selectivity to all the perceived colours, all elaborated from the hue selectivity established in the  retina. A similar mechanism may be in play for all the feature selectivities in the cortex. Cortical inputs can arise from either the left or the right eye but the cortical neurons themselves have a range of ocular dominances. Similarly, for polarity selectivity, cortical inputs may be tuned to light increments or light decrements established by the on and off bipolar cells in the retina but cortical neurons show varying range of selectivity to light increments and decrements. \textbf{In this thesis, we hypothesise that orientation selectivity in V1 too arises from a small number of broadly tuned channels in the retina.}
    
	Perception of spatial vision does not solely involve the orientation of neurons; the spatial frequency tuning of neurons also plays and important role. Further, a complex interaction between the orientation and spatial frequency of a stimulus and the organisation of the receptive field sub-divisions also takes place. \textbf{A secondary aim of this thesis was to examine the interactions between the orientation selectivity, spatial frequency tuning and the linearity of spatial summation of the neurons.}
    
	\section{Approach and Methodology}
	
	We examined orientation selectivity in the visual system of tree shrews and macaques to evaluate the possibility that both the orientation selectivity of individual neurons as well as its organisation in the cortex arise from biased inputs. The visual system of macaques have been studied extensively to understand human vision. The tree shrew is a close ancestor of primates and has an interesting visual system. While in the macaques and cats, ocular dominance and stimulus polarity are organised in columns, in the tree shrews, these features are organised in layers. However, neurons of similar orientation are still grouped in columns in the tree shrew supra granular layers. Further, in the tree shrews, the cortical recipient neurons in layer 4 are not tuned to orientation whereas, layer 2/3 neurons show a similar degree of orientation selectivity as that reported in other species. As a result, there is a unique opportunity to explore the transformation that happens from LGN to layer 4 in cats entirely within the cortex in the tree shrews. 
	
	\section{Chapter Outline}
	\subsubsection{Chapter 1: Literature Review}
	
	This review is organised in three parts.
	
	\paragraph{Visual Pathways}
	First the visual pathway from the retina to through the LGN and to the cortex are described. The transformation of receptive field properties that occurs in each step are also reviewed. An alternate pathway from the retina via the superior colliculus to the visual cortex is also described.
	
	\paragraph{Mechanisms underlying feature selectivity}
	The mechanisms underlying the features studied in this thesis- Orientation selectivity, spatial frequency tuning and linearity of spatial summation are critiqued. Specifically, the feedforward excitatory convergence model of Hubel and Wiesel (1962) proposed to explain orientation selectivity and alternate models of orientation selectivity are evaluated. The alternate LGN driven-recurrent model (ALD-RM; Vidysagar et al., 1996) is described.
	
	\paragraph{Organisation of the feature selectivity in the primary visual cortex}
	The organisation of feature selectivity in the primary visual cortex of the cats, macaques and tree shrews are described.
	
	\subsubsection{Chapter 2: General Methods}
	
	In this chapter, the surgical and anaesthesia procedures, the stimuli used, stimulus presentation and data collection equipment and any data analysis that is commonly used in more than one chapter a described. A brief background on optical imaging of intrinsic signals (OI) is also presented.
	
	\subsubsection{Chapter 3: Radial Bias in the primary visual cortex of macaques}
	
	In this chapter, optical imaging of intrinsic signals was used to record the orientation response of cortical neurons. The band-pass filtering commonly applied while analysing optical imaging data was not performed. We found that this unfiltered signals was tuned to only one orientation, namely the radial orientation. We then recorded from neurons using single electrodes and multi-electrode arrays and found that most of the LFP responses were tuned to the radial orientation whereas the multi-unit activity was tuned a range of different orientations. We suggest the radial orientation represents the inputs to the cortex as a majority of the retinal neurons are tuned to the radial orientation and the unfiltered OI signal and the LFP tend to reflect the pre-synaptic activity rather than post-synaptic activity. Further, we suggest that the cortical neurons elaborate the full gamut of orientation selectivities observed in the cortex from just the radial orientation and pooled responses from neurons tuned to other orientations.
	
	\subsubsection{Chapter 4: Mechanism of orientation selectivity in the tree shrew primary visual cortex}
	
	In this chapter, we recorded the orientation and spatial frequency tuning responses from layer 2/3 and layer 4 of the tree shrew V1. We found that the layer 4 neurons showed broad orientation biases and low-pass spatial frequency tuning. Layer 2/3 neurons showed a bimodal distribution of orientation selectivity where some neurons showed sharp orientation selectivity while others showed broader orientation tuning. A higher proportion of layer 2/3 neurons also showed band-pass spatial frequency tuning when compared to the layer 4 neurons. Further, we also showed that layer 4 and layer 2/3 neurons were tuned to the same orientation while neurons in layer 3c (sub-layer of layer 2/3 located just above layer 4) neurons were tuned to an orientation ~ 60$^o$ away. Layer 4 neurons showed a greater degree of orientation selectivity at higher spatial frequencies. We proposed that the layer 2/3 neurons will fire at these higher spatial frequencies where the layer 4 neurons were sharply tuned to orientation but only found that this was true in 3 of the 18 neuron pairs. These results are placed in the context of the ALD-RM model. The possibility of layer 3c neurons providing inhibition to layer 2/3 neurons are further discussed.
	
	\subsubsection{Chapter 5: Receptive field properties of the tree shrew superior colliculus neurons}
	
	In this chapter, we examined the response properties of neurons in the tree shrew superior colliculus. We measured the orientation selectivity of the SC neurons using bars and gratings and the spatial frequency tuning of neurons in the shrew SC. We compared the results to those obtained from the tree shrew geniculo-striate system and hypothesised that if the neurons were similarly tuned to orientation and spatial frequency in both pathways, then it is likely that they inherited their properties from a common source, namely the retina. We found that a similar proporiton of SC and LGN neurons (Result published in Van Hooser et al., 2013) were biased for orientation and band-pass tuned to spatial frequency. As demonstrated in layer 4 (in chapter 4), the SC neurons were also more tuned for orientation at higher spatial frequencies. Our results indicate that orientation biases observed in the tree shrew LGN and layer 4 are likely to have originated from the same source.
	
	\subsubsection{Chapter 6: Is the tree shrew primary visual cortex a linear filter?}
	
	In this chapter, we recorded from V1 simple cells in the shrew and evaluated if they function as linear filters. It has been controversial whether cortical simple cells function as edge detectors or as fourier analysers when processing visual information. Studies have shown that simple cells in cats and macaques analyse the visual scene in both the space and spatial frequency domains. Here we examined if the shrew V1 neurons functioned as linear filters or as edge detectors and found that simple cells in the tree shrews were far better linear filters than their counterparts in the macaque and cat cortex.
	
	\subsubsection{Chapter 7: General Discussion}
	
	The results obtained from the previous chapters are discussed in the context of generation of feature selectivity as well as the organisation of such feature selectivity in the primary visual cortex. A comparison of the visual systems of the cats, macaques and tree shrews are also undertaken.
\end{document}
