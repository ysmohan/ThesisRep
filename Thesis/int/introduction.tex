\chapter{Introduction}

\subsubsection{Problems and Objectives of the study}

Neurons in the visual system show feature selectivity; they respond optimally to features of stimuli such as the orientation of the stimulus, the 
Over the years, sub-cortical orientation biases have been shown to play a significant role in  two key areas of study in the primary visual cortex. The first is its role in generating sharp orientation selectivity in the cortex and second is its role in generating the cortical architecture. In my thesis, I aim to further characterise the sub-cortical orientation biases and examine their role in visual processing. In the first part of my thesis, I would like to characterise the origin of the biased sub-cortical input to the cortex. There is debate as to exactly when in visual processing the orientation bias observed in the cortical input is generated. Some studies claim that this orientation bias is generated early on in the visual processing: namely the retina. Some others claim that these biases are generated through a mechanism such as excitatory convergence in the cortex. This part probes this question in two ways. 

\subsubsection{Chapter 6}

This chapter examines if there is a preponderance of a particular orientation in the cortical inputs. If the orientation bias in the cortical input is generated by Hubel and Wiesel type excitatory convergence --- where circular LGN receptive fields converge on to a V1 neuron --- we would expect that inputs to the cortex don't show any preferences (i.e. the orientations of the inputs will be randomly distributed.). Many studies however, have shown that RGCs and LGN neurons are preferentially tuned to the radial orientation (the orientation of the line joining the center of the receptive field to the centre of visual field). If the orientation bias in the inputs is derived from the retina instead, then this 
	
	