\chapter{Introduction}
\section{Thesis Summary}
Neurons in the primary visual cortex (V1) show sharp selectivity to
stimulus features such as the orientation \cite{Hubel1962d}, luminance increases and
decreases \cite{Jin2011a}, colour \cite{Hanazawa2000}and spatial and temporal frequencies \cite{Movshon1978}. Broad biases for
these stimuli are however already present sub-cortically and in some
cases, as early as the retina \cite{Bowmaker1980, Levick1980, Levick1982c, Vidyasagar1982, Passaglia2002a, Smith1990b}. For example, cortical inputs can arise
from either the left or the right eye but cortical neurons themselves
have a range of ocular dominances \cite{Hubel1962d} and selectivity for light increments
or light decrements are established by the on and off bipolar cells in
the retina \cite{Kuffler1951} but cortical neurons show varying range of selectivity to
light increments and decrements \cite{Jin2011a}. In this thesis, we examine how visual
neurons elaborate the feature selectivity established in the feedforward
signal.

We studied feature selectivity in anaesthetized tree shrew and macaque
visual neurons. Macaques are evolutionarily close to humans and their
visual system has been studied extensively in an attempt to understand
how we see. The tree shrew is a close ancestor of primates. While in the
macaques and cats, ocular dominance and stimulus polarity are organised
in columns, in the tree shrews, these features are organised in layers \cite{Conley1984}.
However, neurons of similar orientation are still grouped in columns in
the tree shrew supragranular layers \cite{Bosking1997}. Further, in the tree shrews, layer
4 neurons have similar receptive field properties as their LGN
counterparts whereas layer 2/3 neurons show a similar degree of feature
selectivity as that reported in other species \cite{Chisum2003c, VanHooser2013e}. As a result, there is a
unique opportunity to explore entirely within the cortex in the tree
shrews the transformation that happens from LGN to layer 4 in cats
(where most of the studies have been conducted). Hence, feature
selectivity in the tree shrews were also studied.

We used optical imaging of intrinsic signals (OI) to image the response
of cortical neurons to stimuli of different orientations (Chapter 3) to
generate orientation maps. Different spatial filters were applied to
separate the orientation responses of neuronal inputs and outputs.
Single electrode recordings were used to map the topography of the
imaged area. Single and multi-electrode recordings were also used to
record the local field potentials (LFP) and multiunit activity of
neurons.

We used single electrode recordings to record from the primary visual
cortex (Chapter 4 and 6) and the superior colliculus (Chapter 5) of tree
shrews to examine the relationship between orientation selectivity,
spatial frequency tuning and linearity of spatial summation of neurons.

\section{Chapter Outline}

\subsubsection{Chapter 2: Literature Review}


The literature review is organised in three parts.


\paragraph{Visual Pathways}

First the visual pathway from the retina through the LGN to V1 is
described. The transformation of receptive field properties that occurs
at each stage are also reviewed. An alternate pathway from the retina
via the superior colliculus to the visual cortex is also described.


\paragraph{Mechanisms underlying feature selectivity}

The mechanisms underlying the features studied in this thesis-
orientation selectivity and spatial frequency tuning are critiqued.
Specifically, the feedforward excitatory convergence
model\textsuperscript{1} proposed to explain orientation selectivity and
alternate models of orientation selectivity are evaluated. The
asymmetric LGN driven-recurrent model (ALD-RM; \cite{Vidyasagar1996c}) is described.


\paragraph{Organisation of the feature selectivity in the primary visual cortex}

The organisation of feature selectivity in the primary visual cortex of
the cats, macaques and tree shrews are described. Models through which
the columnar organization of the cortex arise are also reviewed. A
recent model that suggests that orientation selectivity arises from a
limited number of broadly tuned channels is described.


\subsubsection{Chapter 3: General Methods}

In this chapter, the surgical and anaesthesia procedures, the stimuli
used, stimulus presentation and data collection equipment and any data
analysis that is commonly used in the chapters are described. A brief
background on optical imaging of intrinsic signals (OI) is also
presented.


\subsubsection{Chapter 4: Radial Bias in the primary visual cortex of macaques}

In this chapter, we examined the orientation biases of the geniculate
inputs to the cortex. Optical imaging of intrinsic signals was used to
record the orientation response of cortical neurons. Besides the
band-pass filtering commonly applied while analysing optical imaging
data, we also made activity maps without any filtering. We found that
this unfiltered signal was tuned to only one orientation, namely the
radial orientation. We then recorded from neurons using single
electrodes and multi-electrode arrays and found that most of the LFP
responses were tuned to the radial orientation whereas the multi-unit
activity was tuned to a range of different orientations.

We suggest that the radial orientation represents the inputs to the
cortex as a majority of the retinal neurons are tuned to the radial
orientation and the unfiltered OI signal and the LFP tend to reflect
largely the pre-synaptic and synaptic activity rather than post-synaptic
activity. Further, we suggest that the cortical neurons elaborate the
full gamut of orientation selectivities observed in the cortex from a
limited number of broadly tuned channels \cite{Vidyasagar2015}.

\subsubsection{Chapter 5: Mechanism of orientation selectivity in the tree shrew primary visual cortex}

In this chapter, the transformation of receptive field properties from
layer 4 to layer 2/3 was studied. We proposed that layer 2/3 neurons
will generate both sharp orientation selectivity and band-pass spatial
frequency tuning from the broadly tuned layer 4 neurons by providing
orientation non-specific inhibition at lower spatial frequencies. We
recorded the orientation and spatial frequency tuning responses from
layer 2/3 and layer 4 of the tree shrew V1. We found that the layer 4
neurons showed broad orientation biases and low-pass spatial frequency
tuning. Layer 2/3 neurons showed a bimodal distribution of orientation
selectivity where some neurons showed sharp orientation selectivity
while others showed broader orientation tuning. Overall, layer 2/3
neurons showed sharper orientation tuning than layer 4 neurons. A higher
proportion of layer 2/3 neurons also showed band-pass spatial frequency
tuning when compared to the layer 4 neurons. Further, we also showed
that layer 4 and layer 2/3 neurons were tuned to the same orientation
while neurons in layer 3c (sub-layer of layer 2/3 located just above
layer 4) neurons were tuned to an orientation ~ 60\textsuperscript{o}
away. Layer 4 neurons showed a greater degree of orientation selectivity
at higher spatial frequencies. We hypothesised that the layer 2/3
neurons fire best at the spatial frequency where the layer 4 neurons are
sharply tuned to orientation but only found that this was true in 3 of
the 18 neuron pairs. The possibility of layer 3c neurons providing
cross-orientation inhibition to layer 2/3 neurons is further discussed.

\subsubsection{Chapter 6: Receptive field properties of the tree shrew superior colliculus neurons}

In this chapter, we examined the orientation and spatial frequency
selectivity of neurons in the superficial layers of tree shrew superior
colliculus (SC). We measured the orientation selectivity of the SC
neurons using bars and gratings and also the spatial frequency tuning of
neurons in the shrew SC. We compared the results to those obtained from
the tree shrew geniculo-striate system and hypothesised that if the
neurons were similarly tuned to orientation and spatial frequency in
both pathways, then it is likely that they inherited their properties
from a common source, namely the retina. We found that a similar
proporiton of SC and LGN neurons (Result published in \cite{VanHooser2013e}) were biased for orientation and low-pass tuned to spatial
frequency. As demonstrated in layer 4 (in chapter 4), the SC neurons
were also more tuned for orientation at higher spatial frequencies. Our
results indicate that orientation biases observed in the tree shrew LGN
and layer 4 are likely to have originated from the same source.

\subsubsection{Chapter 7: Is the tree shrew primary visual cortex an ideal Fourier analyser?}

In this chapter, we recorded the spatial frequency tuning responses from
V1 simple cells in the shrew and evaluated if they function as ideal
Fourier filters, which has been controversial. We tested one of the
central predictions of the patch-by-patch Fourier analysis theory \cite{Robson1975}by
relating peak preferred spatial frequency to bandwidth of spatial
frequency tuning. Our results indicate that though the V1 cells were not
ideal Fourier analyser, they nevertheless behaved as better Fourier
analysers compared to simple cells in macaque and cat cortices.

\subsubsection{Chapter 8: General Discussion}

The results obtained from the previous chapters are discussed in the
context of generation of feature selectivity as well as the organisation
of feature selectivity in the primary visual cortex. A comparison of the
visual systems of the cats, macaques and tree shrews is also undertaken.

\pagebreak
\pagebreak