
	\chapter{Literature Review} 
		\section{Cross-species comparison}
			In my thesis, I am going to compare the laminar organisation of the primary visual cortex, response properties of neurons observed here and focus on the mechanisms involved in the generation of orientation selectivity. Most vision experiments are conducted on cats. But there is a line of evidence which suggests that the way in which orientation selectivity evolves is different between carnivores and primates. In particular, there is a break in the  mouse visual cortex where orientation selectivity is organised in a salt and pepper fashion rather than the usual columnar organisation observed in other species. Here we compare the mechanism of orientation selectivity in three different species, namely cats, tree shrews and macaques. 
			\subsection{Laminar Organisation of neuronal responses.}
				Scope of the section: In my thesis I am going to concentrate mainly on feedforward pathways. In the visual system, information from the eyes is transmitted  from the retina to the lateral geniculate nucleus (LGN). The LGN then transmits information to the primary visual cortex (V1). V1 consists of six layers. LGN input to V1 terminates in layer 4 which then provides input to layer 2/3 of the cortex which in turn provides input to higher visual areas. There is internal feedback from horizontal networks within  areas which will be examined here. Feedback also comes to V1 from the extrastriate areas and V1 itself provides feedback to LGN these are not examined in great detail here. The feedforward pathway is highly conserved but there are certain species specific differences in the responses of neurons in these lamina. The similarities and differences are highlighted below. See figure 1 for a summary of the laminar organisation. I am also going to examine the distribution of orientation and spatial frequency information in these layers.
	
				An image of cat V1 with the layers marked is shown in figure 2.
			\subsubsection{Laminar organisation of neuronal responses in cats area 17}
				 As detailed above, in the cat, LGN inputs terminate at layer 4 of V1. Additionally, LGN also projects to area 18 in the cats. Within layer 4, the C lamina projects to the top and bottom of layer 4 while the A laminae projected throughout layer 4 and to the bottom of layer 3. The C lamina projections essentially "bracket" the A laminae projections (LeVay and Gilbert, 1976). This layer is not as prominent in cats as in other species. Neurons are classified into X or Y type based on the linearity of their response. X cells project to layer 4C and Y celss in layer 4ab (Gilbert and Wiesel, 1979). Lamina A and C= X- input and lamina A1= y input. Lamina B seems to have a lack of x-activity and presence of w-activity.
			 \subsubsection{Laminar organisation of neuronal responses in macaques primary visual cortex}
	 
				 In macaques, the laminar organisation is yet again slightl different. Inputs to the cortex are organised in layer 4 but there is now the added complexity colour information. 
	 
	 
			 \subsubsection{Laminar organisation of neuronal responses in tree shrew primary visual cortex}
	 
				As in cats and macaques, geniculate inputs terminate in layer 4 of the tree shrew cortex (figure 2c). Unlike cats and macaques however, this laminar segregation is between on and off neurons. Layer 4 is further sub-divided into layer 4a and 4b. Layer 4a receives input from the on laminae of the tree shrew LGN and the layer 4b receives input from off laminae of the LGN. There is a 'cleft' segregating the sub-layers on either side of which are present neurons that are tuned to both on and off stimuli. The neurons are also segregated on the basis of ocularity. Neurons that receive contralateral input are found along the outer edges of layer 4, sandwiching neurons that receive binocular inputs. Anatomically, it has been shown that there are parallel inputs from layer 4 to layer 2/3. Inputs from the outer edges of layer 4 are said to project to the lower part of layer 2/3, inputs from the middle of layers 4a and 4b to the middle of layer 2/3 and inputs from the bottom layer of layer 4a and top layer of layer 4b to the topmost layer of layer 2/3. 
	
				Neurons in layer 4 of tree shrews behave rather differently when compared to layer 4 of cats. As mentioned earlier, they are segregated based on their polarity (whether they are on centred or off centred). Further, they also exhibit properties similar to their LGN counterparts in cats. For example, they are broadly tuned to orientation and show a low-pass spatial frequency tuning. Van Hooser et al., 2013 showed that layer 4 neurons in tree shrews are slighty more tuned to orientation than LGN neurons in tree shrew and apart from an attenuation of response to higher temporal frequencies, resemble LGN neurons quite closely. The sharper orientation tuning has been particularly attributed to the edges of layer 4, namely the top of layer 4a and the bottom of layer 4b. Layer 4 neurons are reported as having a similar spatial frequency tuning curve to lgn neurons. Layer 2/3 neurons in tree shrews are however as sharply tuned to both orientation and spatial frequency as the layer 2/3 neurons in cats and monkeys; ie., there are neurons sharply tuned for orientation and organised in columns and the spatial frequency is band pass tuned at the optimum orientation.
	
	
				Anatomically, Fitzpatrick and colleagues suggest that the edges of layer 4 project to the bottom of layer 2/3. However, physiologically, this indicates a connection between a region sharply tuned for orientation to a region that shows orientation selectivity to a lesser extent. This drop in orientation selectivity has been observed between simple and complex cells. Further, other anatomical sources suggest that, similar to the cats and macaques, lgn inputs terminate in the lower parts of layer 2/3. 
	
	\subsection{Orientation selectivity}
	
				The origin of sharp orientation selectivity as observed in the primary visual cortex has long been debated. Hubel and Wiesel when they first described it proposed a model by which this sharp orientation tuning could arise. They suggested that circular LGN receptive fields which are arranged in a row provide input to a cortical neurons which is then tuned to an orientation parallel to that of the LGN receptive fields. This model falls short on multiple accounts. For example it cannot account for such properties as the contrast insensitivity of orientation tuning. It also does not account for the presence of large amounts of inhibition in the primary visual cortex. Finally, it has been demonstrated that neurons in sub cortical areas are already tuned for orientation and Hubel and Wiesel's model does not account for this. 
	


				One model that takes these into account involves sharpening of the orientation biases observed in the V1.In this model, LGN inputs that are already biased for orientation are further sharpened by intra-cortical mechanisms such as orientation non-specific inhibition. This model not only accounts for orientation selectivity but also spatial frequency tuning responses observed.
	
	In cats, LGN neurons show a broad orientation bias and also demonstrate a low pass spatial frequency tuning. Layer 4 neurons on the other hand are sharply tuned to orientation and show a band pass orientation tuning.
	In the LGN, at the optimum spatial frequency, the neurons are broadly tuned to orientation. At higher spatial frequencies they are sharply tuned to orientation almost to the extent observed in V1.
	Cortical neurons receive direct excitatory inputs from the geniculate. They also receive di-synaptic inhibitory input through an inhibitory interneuron. 
	The cortical neuron receives excitatory input that is biased for a certain orientation from the LGN. At the same time, it also gets inhibitory inputs that are either tuned to an orthogonal orientation or broadly tuned to orientation. Then at lower spatial frequencies where none of the geniculate neurons are sharply tuned for orientation, both the excitatory and the inhibitory inputs cancel each other out. But at higher spatial frequencies, the orientation perpendicular to the direction of the inhibition gets no suppression. As a result, the cortical neuron now fires for this orientation.
	
	This mechanism would explain both sharp orientation tuning and spatial frequency tuning of the neuron.
	
	\subsection{Experimental results}
	
	In the orientation selectivity discourse, results from intracellular and extracellular recordings are interpreted differently. Here, I will present a summary of neuronal properties that have been reported and then evaluate to what extent the individual theories explain or rely on this empirical evidence.
	
	\subsubsection{Orientation tuning of subcortical neurons}
	
	While it seems simple enough to measure, the orientation tuning of subcortical neurons is a highly contested research topic. A lot of the earlier studies that examined the receptive field properties of retinal and geniculate neurons assumed circular receptive fields and did not further examine the orientation selctivity in these neurons. Hammond originally showed that the orientation of retinal neurons highly relied on the spatial frequency of the stimulus. This was followed by Levick and Thibos who showed that cat retinal ganglion cells were sharply tuned to orientations at higher spatial frequencies. Since then, orientation tuning at higher spatial frequencies has been demonstrated in cat LGN, macaque retina and LGN. These results suggest that sub cortical neurons are indeed biased for orientation and that this orientation bias is only evident at higher spatial frequencies.
	
	\subsubsection{Orientation tuning of cortical inputs}
	
	\subsubsection{Length tuning of cortical inputs}
	
	\subsubsection{Contrast invariance of orientation selectivity}
	
	
	
	
	
	\section{Sub-cortical orientation biases}
	
	\subsection{Radial Bias in the cortex}
	This will lead to experimental chapter 1
	\subsection{Orientation biases in the superior colliculus}
	This will lead to experimental chapter 2
	\section{Effect of Inhibition}
	This will lead to experimental chapter 3 and 4.
	
	\section{Spatial Frequency Tuning}
	
	\section{Spatial Frequency dependence of Orientation Tuning}
	\section{Superior Colliculus}
	
	The superior colliculus plays an important role in visual processing in the tree shrews. It has two functional sub-divisions, the superficial layers which are involved in form perceptions and the deeper layers which are involved in head movement. The superficial layers receive retinal inputs and projects to extrastriate visual cortex via the pulvinar. They also provide input to other visual areas of the thalamus like the LGN. The visual pathway through the superior colliculus to the temporal cortical region is said to be an alternate to the geniculostriate pathway. If this were indeed the case, then SC could be the dLGN equivalent in this alternate pathway. This would mean that features important in form perception such as orientation selectivity would need to be conveyed to the temporal regions. The orientation input could be derived from striate cortical neurons which are said to generate orientation tuning de novo. However, in studies where V1 was removed, form perception was preserved in animals. This suggests that the basis of orientation selectivity is probably sub-cortical. Infact, orientation biases have been reported in structures as early along the visual pathway as the retina. Cat and macaque retinal ganglion cells are tuned to orientation at higher spatial frequencies. These biases can then be sharpened using intracortical mechanisms.
	
	
	The superior colliculus also known as the tectum in non-mammalian species is a midbrain nucleus that plays an important role in vi is different in different organisms. Broadly it is divided into two functional areas. The dorsal layers of the SC play an important role in form perception. The lower layers are involved in an animal's orienting behaviour. 
	\subsection{Anatomy and projecctions of the superior colliculus}
	\subsection{Response properties of the superior colliculus}
	In cats, neurons in the superficial layers of the superior colliculus are selective to direction of a stimulus. In particular, cells respond best to the direction away from the vertical meridian. SC neurons in other species(eg: frogs) lack this specificity while maintaining the direction selectivity. Other animals like primates seem to entirely lack direction selectivity in the superior colliculus. 
	Orientation selectivity in the superficial layers of the superior colliculus has not been reported in most species. However, a number of recent papers suggest that smaller mammals such as mice demonstrate orientation selectivity in the superficial layers. Studies of orientation selectiivity in the superficial layers of the tree shrew have shown that a small proportion of neurons in the superficial layers are selective to orientation. This study also does not comment on the direction selectivity of neurons in the superficial SC. Direction selectivity is a rare trait in the tree shrew visual system with only a small portion of cells even in the striate cortex demonstrating this property. 
	\subsection{Direction Selectivity}
	29\% of cells in macaque V1 are direction selective 71\% not (Ratio of opt to non-opt $>$ 0.5= non directional; DeValois et al., 1981)





	\section{Literature Review/ Background}
	\subsection{The mammalian primary visual cortex}

	The primary visual cortex (V1 or Area 17) has been studied extensively. Its organising features such as the functional specialisation of individual layers and its columnar architecture have been hailed as representative of the organisation of other sensory cortices. Briefly, feed-forward geniculate (LGN) inputs to V1 terminate in layer 4 and 6. Layer 4 neurons project to the superficial cortical layers (layers 2 and 3) from which extrastriate projections originate. Infragranular layers are believed to be the origins of feedback to the subcortical visual areas (see Douglas \& Martin, 2004 for review). 
	Within this canonical framework, the functional nature of inputs vary.In most species, visual information is segregated on the basis of their functional properties into different pathways. In cats, macaques and tree shrews, the functional segregationdiffers. These differences are briefly examined below (also see Figure 1 for summary).
	In macaques, chromatic and achromatic information is segregated in different pathwaysin their projections from retina to LGN to V1. The magnocellular pathway (M-) transmits achromatic information and the neurons in this pathway respond to luminance changes (Hicks et al., 1983; Kaplan et al., 1990; Dacey, 2001). The parvocellular (P-) (Hicks et al., 1983; Kaplan et al., 1990; Merigan \& Maunsell, 1993) andkoniocellular (K-) (Dacey, 2001; Roy et al., 2009)pathways transmit chromatic information. The  major targets of these projections in macaques are in layer 4Cα, 4Cβand layer 3B of V1 (Casagrande \& Kaas, 1994). This segregation was believed to be maintained even in extrastriate areas (Bullier \& Henry, 1980; Casagrande \& Kaas, 1994). However, there is evidence to suggest that there is considerable overlap in inputs as early as layer 4 (Casagrande \& Kaas, 1994; Callayway, 1998; Vidyasagar et al., 2002).
	In comparison, LGN inputs to V1 in the tree shrew are segregated into ON, OFFand W-cell pathways (Conway \& Schiller, 1983; Conley et al., 1984; Holdefer \& Norton, 1995). ON cells respond to increases in luminance and OFFcells respond to decreases in luminance. The ON, OFF and W-cells terminate in layers 4A, 4B and 3C of V1 respectively (Conley et al., 1984). Layer 4A mostly have on neurons and 4B, mostly off neurons (for review, see Fitzpatrick, 1996).
	In cats, the inputs to V1 segregate differently. X and Y cells of the LGN project to layers 4C and 4A+B respectively (Wilson et al., 1976; LeVay \& Gilbert, 1976). X-cells show a sustained response when presented a stimulus. They also sum signals linearly within their receptive fields. That is, when presented with dark and light stimulus regions over the receptive fieldat the appropriate phase, there is virtually no response as the cell sums the signals from the ON and OFF sub-regionslinearly. Y cells on the other hand sum non-linearly within their receptive fields and they also have a transient response when a stimulus is presented, irrespective of phase (Enroth-Cugell \& Robson., 1966).
	While there are major differences in physiological properties of the different pathways, some similarities have been found. For example, it has been shown that there is some extent of on/off segregation as observed in the tree shrew within the parvocellular layers of the macaque LGN (Schiller \& Malpeli, 1978). It was also originally thought that parvocellular cells were X-cells and magnocellular cells were Y-cells (Dreher et al., 1976). However, this is not entirely the case. While most P-cells are indeed X cells, 75\% of M- cells are also X-cells in macaques (Shapley et al., 1981). Similarly, most neurons in the tree shrew LGN are also X cells, with cells showing non-linear summation only observed in 2 of the 6 layers (Conway \& Schiller, 1983).
	Despite the differences highlighted above, the supragranular layers have similar functional architecture in all three species. Hubel and Wiesel (1962; 1968) first demonstrated the presence of orientation columns in cats and in macaques using electrophysiology.This was also later demonstrated using autoradiographic studies (Hubel et al., 1978).Optical imaging of intrinsic signals showed that orientation in the V1 was organised in columns which converged at pinwheel centres in cats and macaques (Bonhoeffer \& Grinvald, 1991; Bartfeld \& Grinvald, 1992). In the tree shrews, Humphrey and Norton (1980) suggested that orientation columns were organised in elongated columns perpendicular to the V1/V2 border.However, later Bosking et al. (1997) showed using optical imaging of intrinsic signals that orientation columns were organised in a similar fashion to what was observed in macaques and cats. Given this, it may be supposed that while the inputs to V1 in cats, macaques and tree shrew are different, the mechanism through which orientation tuning develops in all three species maybe similar. 


	\subsection{Mechanisms of orientation selectivity}
	Cortical units selectively respond to edges of a narrow range of orientations unlike their LGN counterparts which respond to almost all orientations. Initial insights into this orientation selectivity, were gained from experiments conducted by Hubel and Wiesel. They proposed a theory of excitatory convergence to explain the sharp orientation tuning they observed in cortical simple cells in cats (Hubel \& Wiesel, 1962) and macaques (Hubel\& Wiesel, 1968). They suggested that un-oriented, spatially offset LGN receptive fields arranged collinearly along the long axis of the cortical receptive field, provided inputs to a simple cell giving rise to the classical, elongated receptive fields and sharp orientation tuning observed in cats and macaques (Hubel \& Wiesel, 1962; 1968).While this has been the most prominent theory of orientation selectivity, still retaining support some 50 years after its conception, it is not without its flaws(for review see Vidyasagar et al., 1996; Ferster \& Miller, 2000). For example, while it explains length summation in the cortical neurons, the excitatory convergence model is unable to account for the contrast invariance observed in simple cells (Ferster \& Miller, 2000; Carandini, 2007).The effectof inhibition generated by intracortical interactions have also been implicated in generating the sharp orientation tuning (Creutzfeldt et al., 1974; Sillito, 1975; 1979; Tsumoto et al., 1979; Sillito et al., 1980). In the light of these short comings, many alternative models of orientation selectivityhave been proposed. 
	Other models of orientation tuning involve the role of intracortical circuits in the generation of sharp orientation tuning. These models include processes such as cross-orientation inhibition generated by inhibitory interneurons (Creutzfeldt et al., 1974), iso-orientation facilitation (Douglas et al., 1991; Volgushev et al., 1995), spatially offset excitatory and inhibitory inputs (Heggelund, 1981) and excitatory inputs originating from on and off centred neurons (Soodak, 1987). The models that implicate the intracortical circuits also do not take into account of the weak orientation bias reported in the LGN afferents to the cortical cell. The studies that have shown orientation biases in the afferent input to the cortical cell had assumed that this bias originates from a Hubel and Wiesel type excitatory convergence (for example, see Ferster \& Miller, 2000). Soodak's (1987) model ignores the evidence that in studies where APB (suppresses ON responses in bipolar cells) is administered intravitreally in cats, the orientation tuning of the remaining OFF response often stays unchanged in both cats and monkeys (Schiller, 1982, 1986; Sherk \& Horton, 1984; for review see Schiller, 1992).
	One model of orientation selectivity suggests that the initial orientation tuning is inherited from the orientation biases of LGN neurons(Vidyasagar et al., 1996). According to this model,the bias in the afferent LGN input to a striate simple cell is established by the excitatory input from one or more LGN receptive fields broadly tuned to the same orientation. In line with this model, orientation biases have been demonstrated in the LGN of cats (Vidyasagar \& Urbas, 1982), macaques (Shou \& Leventhal, 1989) and tree shrews (Van Hooser et al., 2013). Once an initial orientation selectivity is established from the LGN input, recurrent excitation and inhibition caused by the extensive horizontal connections in V1; cross-orientation inhibition and non-specific inhibition may all contribute to sharpen orientation tuning (Vidyasagar et al., 1996).
	Inhibition has been shown to play an important role in establishing the sharp orientation tuning observed in cat visual cortex. It was shown that each layer 4 simple cell received monosynaptic excitatory input as well as a disynaptic inhibitory input in the cat (Creutzfeldt \& Ito, 1968; Ferster\& Lindstrom, 1983). Intracellular recordings from V1 neurons where a stimulus moving over the excitatory receptive field often elicited inhibitory post synaptic potentials (IPSP; Creutzfeldt et al., 1974) suggesting that inhibition played an important role in sharpening orientation selectivity. When bicuculline, a GABAA receptor antagonist was applied to V1, a significant reduction in the orientation tuning of several cells was found with orientation tuning abolished entirely in some neurons (Sillito, 1979). In a later study, Sillito et al. (1980) used a more potent GABA antagonist N-methyl bicuculline and found that 9 out of 13 simple cells showed complete loss of orientation selectivity. This mechanism by which the broadly tuned excitatory inputs are sharpened by means of disynaptic inhibition could also explain other cortical properties such as spatial frequency tuning (Vidyasagar \& Heide, 1984; Vidyasagar \& Mueller, 1984; Vidyasagar, 1987) and cortical length response functions (Vidyasagar, 1987; Kuhmann \& Vidyasagar, 2011).
	Sine-wave gratings have been used to study both spatial frequency tuning and orientation tuning in neurons along the visual pathway. When thus examined, cortical cells exhibit band-pass tuning to spatial frequency i.e., they respond to a narrow range of spatial frequencies. Their LGN counterparts on the other hand show a low-pass spatial frequency tuning (Maffei \& Fiorentini, 1973; DeValois et al., 1980; Van Hooser et al.,2013). Further, orientation tuning of neurons are dependent on the spatial frequency of the stimulus used.At lower spatial frequencies, retinal and LGN neurons respond well to gratings of all orientations. At higher spatial frequencies, on the other hand, the orientation selectivity sharpen markedly; i.e., at the non-optimum orientation there is less response to a stimulus when compared to the optimum orientation (Levick \& Thibos, 1980; 1982; Vidyasagar \& Heide, 1984).As LGN neurons do not show orientation specificity at lower spatial frequencies, if these were to drive the cortical inhibitory neurons, the response of cortical neuron studied will be attenuated at lower spatial frequencies through orientation non-specific inhibition. At higher spatial frequencies, the excitatory input that the cortical cells receive from the LGN will be tuned to orientation (Vidyasagar \& Heide, 1984; Vidyasagar, 1987; Kuhlmann \& Vidyasagar, 2011).
	It was originally proposed in the tree shrews that sharp orientation selectivity of layer 2/3 neuronsarose from Hubel and Wiesel (1962) style excitatory convergence of feedforward inputs from layer 4 (Chisum et al., 2003; Mooser et al., 2004) with inhibition exerted by the extensive horizontal interactions leading to further sharpening (as suggested by Ferster \& Miller, 2000). In a recent study by Huang et al (2014), it was shown that the neurons belonging to a particular orientation domain summed their inputs linearly when domains of similar orientation were optogenetically stimulated, questioning the role of inhibition in generating sharp orientation selectivity. However, it must be noted that their optogenetic stimulation would not have activated GABA-ergic neurons in the cortex and any excitation of inhibitory neurons would be post-synaptic which may not have been sufficient to alter the response (Huang et al., 2014). As a result, the role of inhibition in sharpening orientation selectivity in the tree shrew cortex has thus far remained unresolved.
	Layer 4 neurons in tree shrews show broad orientation bias and low pass spatial frequency tuning responses similar totheir LGN counterparts(Chisum et al., 2003; Van Hooser et al., 2013; Scholl et al., 2013). There are extensive short and long range horizontal connections within the tree shrew layer 2/3 which contribute to the orientation response of their target neurons (Bosking et al., 1997; Chisum et al., 2003). Based on this evidence, it may be hypothesised that in tree shrews, sharp orientation tuning observed in the layer 2/3 neurons comes about by the sharpening of orientation biases of the layer 4 neurons through orientation non-specific inhibition similar to the transition from LGN to layer 4 simple cell suggested in cats (Vidyasagar, 1987; Vidyasagar et al., 1996).Experiments 1 and 2 will be conducted to test this hypothesis.
	\subsection{Functional organisation of V1}
	One of the striking features of the primary visual cortex in mammals is the organisation of features such as orientation selectivity and the eye of origin into functional modules (Hubel and Wiesel, 1962; 1968). The organisation of orientation selectivity is such that neurons that have similar orientation preferences are clustered together to form orientation columns (Hubel \& Wiesel, 1962; 1968) with different orientation domains appearing to converge on to “pinwheel centres” (Bonhoeffer and Grinvald, 1991). In the macaque, it has been suggested that the presence of these modules may not be merely functional but rather that there may be a physiological substrate in the cytochrome oxidase (CO) blobs (Livingstone and Hubel, 1982).
	In the primary visual cortex, there are regularly spaced regions of the cortex which stain darkly for the metabolic enzyme, cytochrome oxidase. Termed cytochrome oxidase blobs (CO blobs), these regions indicate areas of high metabolic activity. Layer 4, where subcortical inputs to V1 terminate- hence increasing the metabolic needs of this region, also stains darkly for cytochrome oxidase (Wong-Riley, 1979; Livingstone \& Hubel, 1982). In layer 2/3 of the macaque, cytochrome oxidase blobs are evenly distributed and coincide with the centres of ocular dominance columns as identified by 2-deoxyglucose studies (2 DG). In 2DG studies however, the orientation columns showed up as a mosaic rather than as distinct rows of blobs (Hubel \& Horton, 1981). Optical imaging studies on the other hand suggested that the orientation pinwheels often avoided the centres of the ocular dominance columns but did not necessarily coincide with centres of the CO blobs suggesting that these two systems co-exist independently (Bartfeld and Grinvald, 1992). It is however, possible that the orientation columns and CO blobs have a more complicated relationship. It has also been shown that CO blobs are co-localised with regions of layer 3B which receive LGN input (Livingstone \& Hubel, 1982). It could be that CO blobs coincide with the location of of the broadly tuned thalamic inputs to the cortex. 
	In the cortex, ocular dominance inputs, orientation selective inputs and ON/OFF inputs are segregated in columns (Hubel \& Wiesel, 1962; 1968; Jin et al., 2008, 2011). These columns however, are not separate but partially overlap, as has been described earlier. It has recently been proposed that LGN afferents broadly tuned for all three properties terminate in separate modules in the cortex and the interactions between these could give rise to all the other orientations, ocular dominance (from 1-7) and subfields with different ON and OFF strengths. For example, an ON afferent from the right eye tuned to vertical orientation and an ON afferent from the right eye tuned to horizontal orientation will terminate separately and give rise to all the other orientations in between them (Vidyasagar \& Eysel, 2015, TINS invited review).

	If the orientation tuning of V1 neurons comes about due to the sharpening of subcortical orientation biases, then in order to not lose resolution, orientation information should be coded in a limited number of broadly tuned channels in the retina, similar to what has been observed in colour (Vidyasagar, 1987). Orientation asymmetries have previously been shown in the retina (in cat, Levick \& Thibos, 1982), in the LGN (Shou \& Leventhal, 1989; Vidyasagar \& Urbas, 1989), in the cortex (Chapman \& Bonhoffer, 1998) and at a behavioural level in humans (for example see Orban et al., 1984). Of these, radial orientation bias has in the recent years gained more support than the oblique effect(Sasaki et al., 2006; Swisher et al., 2010). ‘Radial orientation' is the angle theline joining the centre of the receptive field and the foveal centre subtends to the horizontal. While generating orientation maps (as in fig 2C) from optical imaging of intrinsic signals, the response to each individual condition is first high-pass filtered and then low-pass filtered. When the low pas filter was omitted, it was found that the signal (believed to be pre-synaptic in nature) was tuned to the radial orientation (Vidyasagar et al., 2014). The maps thus obtained were termed 'veridical maps'. In figure 2 the veridical orientation maps that have been obtained in our lab are presented. In all the maps generated thus far, the left operculum of macaque V1 was imaged and as a result, the same part of the receptive field was imaged (radial angle corresponding to the second quadrant- between 90o and 180o). In the veridical orientation maps that have been presented, there are also patches (indicated by the black arrows). The relationship between these patches, cytochrome oxidase blobs and the orientation columns observed in the filtered maps will be examined in experiment 3.
	
